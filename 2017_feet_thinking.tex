
\documentclass[compress]{beamer}

%\usepackage{beamerthemesplit}
\usepackage{xmpmulti}

\usepackage{booktabs}
\usepackage{graphicx,float,wrapfig, bbm}
\usepackage{amsfonts, bbold, comment}
\usepackage{mdwlist}
\usepackage{subfigure}
\usepackage{colortbl}

\usepackage{multirow}

\pgfdeclareimage[width=\paperwidth]{mybackground}{../../common/boulder.pdf}

\newcommand{\pos}[1]{{\texttt{#1}}}
\newcommand{\e}[2]{\mathbb{E}_{#1}\left[ #2 \right] }
\newcommand{\ind}[1]{\mathbb{I}\left[ #1 \right] }
\newcommand{\abr}[1]{\textsc{#1} }
\newcommand{\ex}[1]{\mbox{exp}\left\{ #1\right\} }
\newcommand{\g}{\, | \,}
\newcommand{\citename}[1]{#1 }
\newcommand{\fsi}[2]{
\begin{frame}[plain]
\vspace*{-1pt}
\makebox[\linewidth]{\includegraphics[width=\paperwidth]{#1}}
\begin{center}
#2
\end{center}
\end{frame}
}

\newcommand{\gfxs}[2]{
\begin{center}
	\includegraphics[width=#2\linewidth]{simtrans/#1}
\end{center}
}

\newcommand{\gfxq}[2]{
\begin{center}
	\includegraphics[width=#2\linewidth]{qb/#1}
\end{center}
}


\usetheme[bullet=circle,                     % Use circles instead of squares for bullets.
          titleline=true,                    % Show a line below the frame title.
          showdate=true,                     % show the date on the title page
          alternativetitlepage=true,         % Use the fancy title page.
          titlepagelogo=general_figures/culogo,              % Logo for the first page.
          % Logo for the header on first page.
          headerlogo=general_figures/boulder_cs,
          ]{UCBoulder}

\usecolortheme{ucdblack}
\title[Thinking on Your Feet]{Thinking on your Feet: Reinforcement Learning for Incremental
Language Tasks}
\author{ Jordan Boyd-Graber}
\date{Fall 2015}

\institute[Boulder] % (optional, but mostly needed)
{University of Colorado Boulder}

\AtBeginSection[] % "Beamer, do the following at the start of every section"
{ \begin{frame} \frametitle{Outline} % make a frame titled "Outline"
\tableofcontents[currentsection] % show TOC and highlight current section
\end{frame} }

\begin{document}

\frame{
\titlepage
\tiny
}


\begin{frame}{Nuremburg Trials}

\begin{columns}

\column{.5\linewidth}

    \gfxs{nuremberg_trials}{1.0}

\column{.5\linewidth}

    \begin{itemize}
        \item Dozens of defendants
        \item Judges from four nations (three languages)
\pause
        \item Status quo: speak, then translate
\pause
        \item After Nuremberg, simultaneous translations became the
          norm
\pause
        \item Long wait $\rightarrow$ bad conversation
     \end{itemize}

\end{columns}

\end{frame}



\begin{frame}{Algorithms that think on their feet}

\begin{columns}

  \column{.65\linewidth}
  \begin{itemize}
     \item Algorithms that process examples \emph{one word at a time}
       \begin{itemize}
         \item Simultaneous machine translation
         \item Trivia games
       \end{itemize}
      \item Similar structure
        \begin{itemize}
          \item Prediction
          \item Policy
        \end{itemize}
  \end{itemize}

  \column{.3\linewidth}

  \gfxs{nuremberg_translators}{.7}
  \gfxq{quizbowl}{.7}

\end{columns}

\end{frame}



\begin{frame}{Very Different Tasks, Common Foundation}

\begin{columns}
\column{.5\linewidth}
  \gfxs{pancake_robot}{.4}
  \begin{center}
  \end{center}
  
  \column{.5\linewidth}
  \begin{itemize}
    \item Reinforcement learning
    \item Machine learning / robotics framework
    \item Underused in NLP
  \end{itemize}
\end{columns}

\vspace{-.5cm}

  \begin{center}
\begin{tabular}{ccc}
\hline
  & QA & MT \\
\hline
{\bf State} & Words Seen & Foreign Words Seen \\
{\bf Reward} & Answer Accuracy & Translation Quality \\
{\bf Actions} & Answer / Wait & Translate / Wait \\
\hline
\end{tabular}

  \end{center}

\end{frame}

\section{Quiz Bowl}

\begin{frame}
	\frametitle{Humans doing Incremental Classification}
	\begin{columns}

	\column{.5\linewidth}
	\begin{itemize}
		\item Game called ``quiz bowl''
		\item Two teams play each other
		\begin{itemize}
			\item Moderator reads a question
			\item When a team knows the answer, they signal (``buzz'' in)
			\item If right, they get points; otherwise, rest of the question is read to the other team
		\end{itemize}
		\item Hundreds of teams in the US alone
                \only<2>{\item Example \dots}
	\end{itemize}

	\column{.5\linewidth}
	\includegraphics{qb/quizbowl}

	\end{columns}

\end{frame}

\begin{frame}[t]
	\frametitle{Sample Question}

        The Swiss-Italian architect Pietro Antonio Solari
        \only<2->{built several fortified towers in this city, which
          often vied for power with its northern rival Tver. A ruler
          of this city prevailed in the} \only<3->{Great Stand on the
          Ugra River. A prince from this city was nicknamed for
          winning a battle on the} \only<4->{Don river. Partly because
          a ruler of this city married} \only<5->{Sophia Palaiologina,
          the niece of the last Byzantine Emperor, this city styled
          itself the} \only<6->{``Third Rome'' after the fall of
          Constantinople. Another prince of this city stopped paying
          tribute to the} \only<7->{Mongols in 1476, ending the
          ``Tatar yoke.''} \only<8->{The Grand Duchy headquartered in
          this city came to an end in 1547 with the ascension of}
        \only<9->{ Ivan IV, who made it his capital. For 10 points,
          name this city where Ivan III renovated the
          Kremlin,} \only<10->{the capital of Russia.}\\
        \vspace{.5cm} \only<11->{ {\bf Moscow} (Moskva / Muscovy)}

\only<12->{
\vspace{-6cm}

\begin{block}{Faster = Smarter}

  \begin{enumerate}
    \item Colorado School of Mines
    \item Cornell University
    \item Brigham Young University
    \item California Institute of Technology
    \item Peking University
    \item Harvey Mudd College
    \item Darmstadt University
    \item University of Colorado
  \end{enumerate}


\end{block}
}



\end{frame}


\begin{frame}
	\frametitle{Humans doing Incremental Classification}

	\begin{columns}
		\column{.5\linewidth}

		\includegraphics[width=1.0\linewidth]{qb/jeopardy}


		\column{.5\linewidth}
		\begin{itemize}
			\item This is {\bf not} Jeopardy \cite{ferruci-10}
			\item There are buzzers, but players only buzz
                          at the question end
			\item Doesn't discriminate knowledge
			\item Quiz bowl is pyramidal
                        \item Watson must decide to answer {\bf once}, after
                          complete question
                        \item Quiz Bowl: decide after each word
		\end{itemize}

	\end{columns}

\end{frame}


\begin{frame}{How to approach this problem \dots}

    \only<1>{
  \begin{columns}
    \column{.5\linewidth}
    \gfxq{guess}{0.8}
    \column{.5\linewidth}
    \gfxq{buzzer}{0.8}
  \end{columns}
}
\only<2>{
   \gfxq{guess}{0.5}
}
\end{frame}



\begin{frame}{}

  \begin{columns}
    \column{.5\linewidth}
        \includegraphics[width=0.7\linewidth]{general_figures/mohit}
    \column{.5\linewidth}
        \begin{block}{ {\bf \href{http://cs.colorado.edu/~jbg//docs/2014_emnlp_qb_rnn.pdf}{A Neural Network for Factoid Question Answering over Paragraphs}}}
\underline{\href{http://cs.umd.edu/~miyyer/}{Mohit Iyyer}}, {\bf Jordan Boyd-Graber}, Leonardo Claudino, Richard Socher, and Hal {Daum\'{e} III}.  \emph{Empirical Methods in Natural Language Processing}, 2014
        \end{block}

        \begin{block}{ {\bf \href{file:///Users/jbg/public_html/docs/2015_acl_dan.pdf}{Deep Unordered Composition Rivals Syntactic Methods for Text Classification}}}
\underline{\href{http://cs.umd.edu/~miyyer/}{Mohit Iyyer}}, Varun
Manjunatha, {\bf Jordan Boyd-Graber} and Hal {Daum\'{e} III}.  \emph{Empirical Methods in Natural Language Processing}, 2014
        \end{block}

  \end{columns}
\end{frame}
\begin{frame}{Vector Space Model}

  \only<1>{\gfxq{unigram_models_0}{.8}}
  \only<2>{\gfxq{unigram_models_1}{.8}}
  \only<3>{\gfxq{unigram_models_2}{.8}}
  \only<4>{\gfxq{unigram_models_3}{.8}}
  \only<5>{\gfxq{unigram_models_4}{.8}}
  \only<6>{\gfxq{unigram_models_5}{.8}}
  \only<7>{\gfxq{unigram_models_6}{.8}}
  \only<8>{\gfxq{unigram_models_7}{.8}}
  \only<9>{\gfxq{unigram_models_8}{.8}}


\end{frame}



\begin{frame}{How can we do better?}

  \begin{itemize}
    \item Use relationship between questions (``China'' and
      ``Taiwan'')
    \item Use learned features and dimensions, not the words we start with
  \end{itemize}

\end{frame}



\begin{frame}{Deep Averaging Networks}

  \only<1>{\gfxq{dan_1}{.8}}
  \only<2>{\gfxq{dan_2}{.6}}
  \only<3>{\gfxq{dan_3}{.6}}
  \only<4>{\gfxq{dan_4}{.6}}

\end{frame}




\begin{frame}{Training}

  \begin{columns}
    \column{.5\linewidth}
      \begin{itemize}
        \item Initialize embeddings from \textsc{word2vec}
        \item Randomly initialize composition matrices
        \item Update using \textsc{warp}
          \begin{itemize}
            \item Randomly choose an instance
            \only<2->{\item Look where it lands}
            \only<4->{\item Has a correct answer}
            \only<5->{\item Wrong answers may be closer}
            \only<6->{\item Push away wrong answers
            \item Bring correct answers closer}
          \end{itemize}
      \end{itemize}

    \column{.5\linewidth}

      \only<1>{\gfxq{warp_training_5}{.8}}
      \only<2>{\gfxq{warp_training_4}{.8}}
      \only<3>{\gfxq{warp_training_3}{.8}}
      \only<4>{\gfxq{warp_training_2}{.8}}
      \only<5>{\gfxq{warp_training_1}{.8}}
      \only<6>{\gfxq{warp_training_0}{.8}}
  \end{columns}

\end{frame}

\begin{frame}{Performance (Sentiment Analysis)}


\footnotesize
\begin{center}
\begin{tabular}{cccccc}
\toprule
Model & RT & SST & SST & IMDB & Time \\
& & fine & bin & & (s)\\
\midrule
\footnotesize DAN & 80.3 & 47.7 & 86.3 & 89.4 & 136\\
\midrule
\footnotesize NBOW-RAND & 76.2 & 42.3 & 81.4 & 88.9 & 91 \\
\footnotesize NBOW & 79.0 & 43.6 & 83.6 & 89.0 & 91 \\
\footnotesize BiNB & --- & 41.9 & 83.1 & --- & ---\\
\footnotesize NBSVM-bi & 79.4 & --- & --- & 91.2 & ---\\
\midrule
\footnotesize RecNN$^*$ & 77.7 & 43.2 & 82.4 & --- & --- \\
\footnotesize RecNTN$^*$ & --- & 45.7 & 85.4 & --- & --- \\
\footnotesize DRecNN & --- & 49.8 & 86.6 & --- & 431\\
\footnotesize TreeLSTM & --- & \bf 50.6 & 86.9 & --- & --- \\
\footnotesize DCNN$^*$ & --- & 48.5 & 86.9 & 89.4 & ---\\
\footnotesize PVEC$^*$ & --- & 48.7 & 87.8 & \bf 92.6 & --- \\
\footnotesize CNN-MC & \bf 81.1 & 47.4 & \bf 88.1 & --- & 2,452 \\
\footnotesize WRRBM$^*$ & --- & --- & --- & 89.2 & ---\\
\bottomrule
\end{tabular}
\end{center}
\end{frame}


\begin{frame}{Embedding}

  \gfxq{embedding}{1.0}

\end{frame}

\begin{frame}{Helping Understanding}


  \begin{center}
    \includegraphics[width=.8\linewidth]{qb/lucy_and_arthur}
  \end{center}

  Digital humanities needs plot summaries for distant
  reading~\cite{iyyer-16} (NAACL 2016 Best Paper) 

\end{frame}

\begin{frame}{How to approach this problem \dots}

    \only<1>{
  \begin{columns}
    \column{.5\linewidth}
    \gfxq{guess}{0.8}
    \column{.5\linewidth}
    \gfxq{buzzer}{0.8}
  \end{columns}
}
\only<2>{
   \gfxq{buzzer}{0.5}
}
\end{frame}


\begin{frame}{}

  \begin{columns}
    \column{.5\linewidth}
        \includegraphics[width=0.7\linewidth]{general_figures/hehe}
    \column{.5\linewidth}
        \begin{block}{{\bf
              \href{http://cs.colorado.edu/~jbg//docs/qb_emnlp_2012.pdf}{Besting
                the Quiz Master: Crowdsourcing Incremental
                Classification Games}}}

          {\bf Jordan Boyd-Graber}, He He, and Hal {Daum\'{e} III}. \emph{Empirical Methods in Natural Language Processing}, 2012
        \end{block}

        \begin{block}{{\bf
              \href{http://www.cs.colorado.edu/~jbg/docs/2016_icml_opponent.pdf}{Opponent Modeling in Deep Reinforcement Learning}}}

          He He, {\bf Jordan Boyd-Graber}, Kevin Kwok, and Hal
          {Daum\'{e} III}. \emph{International Conference of Machine Learning}, 2016
        \end{block}

  \end{columns}
\end{frame}


\begin{frame}
\frametitle{Interface}

\begin{columns}

	\column{0.5\linewidth}

	\begin{center}
		\includegraphics[width=0.8\linewidth]{qb/screenshot}
	\end{center}

	\column{0.5\linewidth}


	\only<2>{
	\begin{itemize}
		\item 7000 questions: first day
		\item 43000 questions: two weeks
		\item 461 unique users
                \item Imitated \dots
	\end{itemize}
        \gfxq{protobowl}{.8}
	}



\end{columns}
\end{frame}




\begin{frame}[plain]


\only<4->{\vspace{-.5cm}}

  \begin{columns}[T]
    \column{.3\linewidth}

    \only<1->{ \includegraphics[width=2\linewidth]{qb/feature_ex_l_1} \\ }
    \vspace{.5cm}
    \only<4->{ \includegraphics[width=2\linewidth]{qb/feature_ex_l_2}  \\ }
    \vspace{.5cm}
    \only<7->{ \includegraphics[width=2\linewidth]{qb/feature_ex_l_3}  \\ }


    \column{.68\linewidth}
    \vspace{-.5cm}
    \only<2->{ \includegraphics[width=.85\linewidth]{qb/feature_ex_r_1} \\ }
    \only<3->{ \vspace{-.5cm} \hspace{.5cm} \includegraphics[width=.1\linewidth]{qb/feature_ex_wait}  \\ }
    \only<5->{ \includegraphics[width=\linewidth]{qb/feature_ex_r_2} \\ }
    \only<6->{ \vspace{-.5cm} \hspace{.5cm}\includegraphics[width=.1\linewidth]{qb/feature_ex_wait}  \\ }
    \only<8->{ \includegraphics[width=\linewidth]{qb/feature_ex_r_3} \\ }
    \only<9->{ \vspace{-.5cm} \hspace{.5cm} \includegraphics[width=.1\linewidth]{qb/feature_ex_buzz}  \\ }
    \only<9->{Answer: {\bf Julius Caesar}}
  \end{columns}

\end{frame}

\begin{frame}{How can this fail?}

  \only<1>{\gfxq{opponent_fail1}{.8}}
  \only<2>{\gfxq{opponent_fail2}{.8}}
  \only<3>{\gfxq{opponent_fail3}{.8}}
  \only<4>{\gfxq{opponent_fail4}{.8}}
  \only<5>{\gfxq{opponent_fail5}{.8}}
  \only<6>{\gfxq{opponent_fail6}{.8}}
  \only<7>{\gfxq{opponent_fail7}{.8}}

\end{frame}

\begin{frame}{Can we do better?}

  \only<1>{\gfxq{dqn_overview2}{.8}}
  \only<2>{\gfxq{dqn_overview3}{.8}}
  \only<3>{\gfxq{dqn_overview4}{.8}}

\end{frame}

\begin{frame}{Not all opponents are equal}

  \gfxq{player_profile}{.9}

  \pause
  Varies by category!

\end{frame}

\begin{frame}{Comparing Models}


\end{frame}

\begin{frame}{Add more features?}

  \gfxq{dron-concat}{.8}

\end{frame}

\begin{frame}{Mixture of Experts}

  \only<1>{\gfxq{dron-moe}{.8}}
  \only<2>{\gfxq{dron-moe2}{.8}}

\end{frame}

\begin{frame}{Reward}

  \gfxq{dqn_results}{.5}

\end{frame}

\begin{frame}{Reward: Closer Look}

  \only<1>{\gfxq{reward1}{.8}}
  \only<2>{\gfxq{reward2}{.8}}
  \only<3>{\gfxq{reward3}{.8}}
  \only<4>{\gfxq{reward4}{.8}}
\end{frame}

\begin{frame}{Error Analysis}

  \only<1>{\gfxq{error1}{.8}}
  \only<2>{\gfxq{error2}{.8}}
  \only<3>{\gfxq{error3}{.8}}
  \only<4>{\gfxq{error4}{.8}}
  \only<5>{\gfxq{error5}{.8}}
  \only<6>{\gfxq{error6}{.8}}
  \only<7>{\gfxq{error7}{.8}}

\end{frame}


\begin{frame}{Experiment 1}

		\begin{columns}
			\column{.25\linewidth}
				\gfxq{colby_jeo}{1.0}
                                Colby Burnett:
                                \$375,000
			\column{.25\linewidth}
				\gfxq{ben_jeo}{1.0}
                                Ben Ingram:
                                \$427,534
			\column{.25\linewidth}
				\gfxq{alex_jeo}{1.0}
                                Alex Jacobs: \$151,802
			\column{.25\linewidth}
				\gfxq{kristin_jeo}{1.0}
                                Kristin Sausville: \$95,201
		\end{columns}

                \pause


                \begin{center}
                End result: 200-200 tie!
                \end{center}

\end{frame}

\fsi{qb/hsnct1}{}
\fsi{qb/jennings}{23. October 2015, Seattle}
\fsi{qb/jennings_handshake}{300-160}

\fsi{qb/nasat}{Humans 345-145}
\fsi{qb/hsnct_2016}{Humans 190-155}

\fsi{qb/nips_demo}{Can beat most ML researchers (NIPS 2015 Best Demo) }

\begin{frame}{Where we have problems}

\only<1-2>{
\begin{block}{Out of Date}
Although he won the California primary in 2000, he distanced himself
from fellow reform presidential candidate Pat Buchanan by comparing
him to Attila the Hun. After being called a jackass, he prompted
Lindsey Graham to destroy his phone by giving out his number during a
speech. The slogan (*) Make America Great Again has been used by this
politician, who claimed he didn't like people who were captured as a
slight to John McCain and kicked off his 2016 presidential bid with
some inflammatory remarks about Mexicans. For 10 points, name this
Republican candidate and real estate mogul.
\end{block} }
\only<2>{{\bf Chris Christie?}}

\only<3-4>{
\begin{block}{Out of Touch}
  This singer recently cancelled the Great Escape Tour, and, in one
  song, she claims that she will be ``Eating crumpets with the sailors
  / On acres without the neighbors.'' She collaborated with Jennifer
  (*) Hudson on the song ``Trouble,'' which was issued in her album
  update Reclassified. This artist of ``Change Your Life'' was
  inspired by scenes from the movie Clueless to make the music video
  for a song in which she collaborated with Charli XCX. For 10 points,
  name this Australian rapper whose album The New Classic contained
  ``Fancy.''
\end{block} }
\only<4>{{\bf Bruce Springsteen?}}


\end{frame}


\begin{frame}[plain]
\gfxq{seattle_crowd}{.5}
\gfxq{chicago_crowd}{.5}
\end{frame}

\fsi{qb/boring_dot_products}{}


\section{Simultaneous Translation}

\begin{frame}{Simultaneous translation is the norm}

  \begin{columns}
    \column{.5\linewidth}
       \gfxs{nuremberg_translators}{.9}
    \column{.5\linewidth}
       \begin{itemize}
         \item Rigorous training
         \item Technological sophistication
         \item Long way from ``sentence at a time''
       \end{itemize}
  \end{columns}

\end{frame}

\begin{frame}{Why simultaneous translation really hard is}

  \begin{columns}
    \column{.5\linewidth}
      \begin{itemize}
        \item Many languages are \textsc{sov}
        \item \alert<2>{German}, Japanese, Farsi, Korean,
          \alert<3>{Yiddish}
        \item<4-> First learned system for simulatenous translation
      \end{itemize}
    \column{.5\linewidth}
      \gfxs{yoda}{.6}
  \end{columns}

  \centering

\only<4-5>{
\vspace{1cm}

\begin{tabular}{c@{ }c@{ }c@{ }c@{ }c@{ }c@{ }c@{ }l}
ich & bin & mit & dem & Zug & nach & Ulm & {\bf gefahren} \\
I & am & with & the & train & to & Ulm & {\bf traveled} \\
\hline
I & \multicolumn{6}{c}{\emph{(\dots\dots waiting\dots\dots)}} & {\bf traveled} by train to Ulm \\
\end{tabular}
}

\only<6>{
  \gfxs{en_ja_example}{.8}
}

\end{frame}


\begin{frame}{}

  \begin{columns}
    \column{.5\linewidth}
        \includegraphics[width=0.9\linewidth]{general_figures/alvin}
    \column{.5\linewidth}
        \begin{block}{ {\bf \href{http://cs.colorado.edu/~jbg//docs/2014_emnlp_simtrans.pdf}{Don't Until the Final Verb Wait: Reinforcement Learning for Simultaneous Machine Translation}}}
\underline{\href{http://www.umiacs.umd.edu/~alvin/}{Alvin Grissom II}}, {\bf Jordan Boyd-Graber}, He He, John Morgan, and Hal {Daum\'{e} III}.  \emph{Empirical Methods in Natural Language Processing}, 2014
        \end{block}
  \end{columns}
\end{frame}

\begin{frame}{Solution: Predicting the Verb}

\begin{columns}

\column{.5\linewidth}
  \begin{itemize}
    \item If we can figure out the verb, we can ``complete'' the
      sentence
    \item This is provided by language models that can predict the
      next word in a sentence
    \item Instead, we'll predict the verb
  \end{itemize}

\column{.5\linewidth}

\gfxs{autocomplete}{.8}

\end{columns}

\end{frame}


\begin{frame}{Language Models of Verbs}

  \only<1>{\gfxs{verb_corpus_1}{.9}}
  \only<2>{\gfxs{verb_corpus_2}{.9}}
  \only<3>{\gfxs{verb_corpus_3}{.9}}

\end{frame}

\begin{frame}{Predicting the Verb}
\begin{itemize}
  \item Build language model for every verb
  \item Then, for any input text $x$ we can make a prediction of the verb
\begin{equation}
  \arg\max_v p(v) \prod_{i=1}^t p(x_i \g v, x_{i-n+1:i-1})
\end{equation}
\pause
  \item Most of these predictions will be totally wrong (18\%
    accuracy) \dots
  \item leading to horrible translations
\end{itemize}
\end{frame}

\begin{frame}{States and Actions}

  \begin{itemize}
    \item State
      \begin{itemize}
        \item The words we've seen
        \item Predictions
      \end{itemize}
      \pause
    \item Actions (more complicated than Quiz Bowl)
      \begin{itemize}
        \item Wait
        \item \alert<3>{Predict Verb and Translate}
        \item \alert<3>{Commit to Translation}
      \end{itemize}
    \end{itemize}
\end{frame}

\begin{frame}{Translations}
  \begin{itemize}
    \item Assume a ``black box''
    \item German in, English out
  \end{itemize}

\end{frame}

\begin{frame}{Consensus Translation}

  \only<1>{\begin{center}
``German in, English out'' black box
      \end{center}}

  \only<2>{\gfxs{consensus_0}{.8}}
  \only<3>{\gfxs{consensus_1}{.8}}
  \only<4>{\gfxs{consensus_2}{.8}}
  \only<5>{\gfxs{consensus_3}{.8}}
  \only<6>{\gfxs{consensus_4}{.8}}

\end{frame}

\begin{frame}{Scoring one Translation}
  \begin{center}
    Bilingual Evaluation Understudy (BLEU)
\end{center}
  \only<1>{\gfxs{bleu_ex}{.8}}
  \only<2>{\gfxs{bleu_correlation}{.6}}
\end{frame}


\begin{frame}{Scoring a series of Translations}
  \begin{center}
    Bilingual Evaluation Understudy (BLEU)
\end{center}
  \only<1>{\gfxs{integral_0}{.95}}
  \only<2>{\gfxs{integral_1}{.95}}
\end{frame}



\begin{frame}{Comparing Policies}

  \only<1-7>{\vspace{4.75cm}}
  \only<8-14>{\vspace{3.3cm}}
  \only<15-21>{\vspace{2cm}}

  \only<1,8,15,22>{\gfxs{reward_example_0}{.8}}
  \only<2,9,16,23>{\gfxs{reward_example_1}{.8}}
  \only<3,10,17,24>{\gfxs{reward_example_2}{.8}}
  \only<4,11,18,25>{\gfxs{reward_example_3}{.8}}
  \only<5,12,19,26>{\gfxs{reward_example_4}{.8}}
  \only<6,13,20,27>{\gfxs{reward_example_5}{.8}}
  \only<7,14,21,28>{\gfxs{reward_example_6}{.8}}
\end{frame}




\begin{frame}{Imitation Learning}

  \begin{columns}
    \column{.5\linewidth}
       \gfxs{imitation_fold}{.8}
       \gfxs{imitation_drive}{.8}
    \column{.5\linewidth}
    \begin{itemize}
      \item Given all the predictions that we make (and the resulting
        translations) \dots
      \item Discover the optimal in hindsight policies
      \item Goal: Teach our algorithm to think on its feet
      \item Challenge: Represent states in a way that will generalize
    \end{itemize}

  \end{columns}

\end{frame}


\begin{frame}{How do we find a good policy?}

  \only<1>{\gfxs{searn_7}{.8}}
  \only<2>{\gfxs{searn_6}{.8}}
  \only<3>{\gfxs{searn_5}{.8}}
  \only<4>{\gfxs{searn_4}{.8}}
  \only<5>{\gfxs{searn_3}{.8}}
  \only<6>{\gfxs{searn_2}{.8}}
  \only<7>{\gfxs{searn_1}{.8}}
\end{frame}


\begin{frame}{How do we find a good policy?}

  \begin{itemize}
    \item Find optimal policies through dynamic programming $\pi_0
      \equiv \pi*$
    \item Represent states $s$ through a feature vector $\vec f(s)$
      \pause
    \item Until convergence:
      \begin{itemize}
        \item Generate examples of state action pairs: $(\pi_t(s), s)$
        \item Create a classifier that maps states to actions (an
          apprentice policy) $h_t: f(s) \mapsto A$ \only<3->{(Loss of
            classifier is the negative reward)}
    \item Interpolate learned classifier $\pi_{t+1} = \lambda \pi_t +
      (1-\lambda) h_t$
  \end{itemize}
  \end{itemize}

\only<4->{
  \begin{center}
    \textsc{searn}: \underline{Se}arching to L\underline{earn}
    (Daum\'e \& Marcu, 2006)
    \end{center}
}
\end{frame}


\begin{frame}{Experimental Setup}

  \begin{itemize}
    \item \textsc{denews} corpus
      \begin{itemize}
        \item News snippets
        \item Some formulaic
        \item Some surprising
      \end{itemize}
    \item Only verb-final sentences
    \item Some compromises with translation system (to make sure verbs appeared)
  \end{itemize}

\end{frame}



\begin{frame}{Comparing Policies}

  \gfxs{cummulative}{.6}

\end{frame}

\begin{frame}{Learned Policy with Accumulated Reward}

  \only<1>{\gfxs{compare_line_batch}{.8}}
  \only<2>{\gfxs{compare_line_batch_monotone}{.8}}
  \only<3>{\gfxs{compare_line_batch_monotone_opt}{.8}}
  \only<4>{\gfxs{compare_line_all}{.8}}
\end{frame}


\begin{frame}{Example Sentence}

  \gfxs{ex_imperfect}{.7}

  \pause

  But we're still at the mercy of our translation black box

\end{frame}


\begin{frame}{}

  \begin{columns}
    \column{.5\linewidth}
        \includegraphics[width=0.8\linewidth]{general_figures/hehe}
    \column{.5\linewidth}
        \begin{block}{ {\bf \href{http://cs.colorado.edu/~jbg//docs/2015_emnlp_rewrite.pdf}{Syntax-based Rewriting for Simultaneous Machine Translation}}}
He He, Alvin Grissom II, {\bf Jordan Boyd-Graber}, and Hal {Daum\'{e} III}.  \emph{Empirical Methods in Natural Language Processing}, 2015
        \end{block}
  \end{columns}
\end{frame}

\begin{frame}{How can we make MT better?}

  \gfxs{en_ja_example}{.9}

  \pause

  \begin{itemize}
    \item In a perfect world, we'd use interpretation data
      \pause
      \item Very rare, and low quality in other ways
      \pause
    \item But can we \emph{rewrite} batch translations?
  \end{itemize}

\end{frame}

\begin{frame}{Are there enough opportunities?}


\begin{center}
\begin{tabular}{lcccc}
\toprule
& \alert<2>{verb} & \alert<3>{voice} & \alert<4>{noun} & \alert<5>{clauses} \\
\midrule
Applicable \% & \only<2->{39.9} & \only<3->{50.0} & \only<4->{26.4} & \only<5->{4.8} \\
Accepted \% & \only<2->{22.5} & \only<3->{24.0} & \only<4->{51.2} & \only<5->{38.4} \\
\bottomrule
\end{tabular}
\end{center}

\only<2>{
\begin{itemize*}
\item[O:] {\bf They announced} that the president will restructure the division.
\item[R:] The president will restructure the division, {\bf they announced}.
\end{itemize*}
}


\only<3>{
  \begin{columns}
    \column{.4\linewidth}

    \gfxs{rewrite_input}{.9}
    \column{.55\linewidth}
    \vspace{-.5cm}
    \gfxs{rewrite_transform}{.9}
   \end{columns}
}

\only<4>{
\begin{itemize*}
  \item[O:] the e-mail server of Clinton
  \item[R:] Clinton's e-mail server
\end{itemize*}
}

\only<5>{
\begin{itemize*}
\item[O:] \pos{S}$_1$ \pos{conj} \pos{S}$_2$: We should march {\bf because} winter is coming.
\item[O:] \pos{conj} \pos{S}$_2$, \pos{S}$_1$: {\bf Because} winter is
  coming, we should march.
\vspace{1cm}
\item[R:] \pos{S}$_2$, \pos{conj'} \pos{S}$_1$: Winter is coming, {\bf because of this}, we should march.
\end{itemize*}
}


\end{frame}

\begin{frame}{}

\gfxs{rewrite_eval}{.8}
\pause
We rewrite 32.2\% of
sentences, reducing the delay from 9.9 words/seg to 6.3 words/seg per
segment for rewritten sentences and from 7.8 words/seg to 6.7 words/seg overall.

\end{frame}

\begin{frame}{How good are the translations?}

  \gfxs{tradeoff-rw-bleu}{.7}

\begin{center}
Aggressiveness based on different right probability thresholds
\cite{fujita-13}
\end{center}

\end{frame}

\begin{frame}{Why does the quality improve?}

\begin{center}
\begin{tabular}{ccccc}
\toprule
& \multicolumn{3}{c}{Translation} & \\
\cmidrule{2-4}
& \abr{gd} & \abr{rw} & \abr{rw+gd} & Gold ref \\
\midrule
\# of verbs & 1971 & 2050 & {\bf 2224} & 2731 \\
\bottomrule
\end{tabular}
\end{center}

\end{frame}

\begin{frame}{Future Steps}

  \begin{itemize}
    \item Richer translation model
    \item Paraphrase database
    \item Better reward (e.g., MEANT)
    \item Verb prediction through argument structure
  \end{itemize}

\end{frame}

\section{Conclusions}

\begin{frame}{Conclusion}

  \begin{itemize}
    \item Algorithms that think on their feet
      \begin{itemize}
        \item First learned system for simultaneous translation
        \item Moving being hand-engineered or bag of words features
          for question answering
      \end{itemize}
    \item Interaction between prediction and confidence
    \item Imitation of what humans do
  \end{itemize}

\end{frame}

\frame{
  \frametitle{But wait, there's more!}

  \vspace{-.5cm}

\begin{columns}



  \column{.5\linewidth}

   \begin{block}{Computational Social Science}
     \centering
     \includegraphics[width=0.9\linewidth]{teaparty/figures/framing} \\
     \cite{nguyen-13b,nguyen-15}
   \end{block}


    \begin{block}{Interactive Machine Learning}
     \centering
        \includegraphics[width=0.4\linewidth]{interactive_topic_models/new_interface} \\
       \cite{boyd-graber-06b,ma-09,nikolova-09}
    \end{block}


  \column{.5\linewidth}


    \begin{block}{Multilingual Topic Models}
      \begin{center}
        \begin{large}
          $p_{\mbox{topic}}(e | f)$ \\
         \end{large}
      \cite{eidelman-12,hu-14}
       \end{center}
    \vspace{-.3cm}
    \end{block}


    \begin{block}{Sentiment / Internal State}
    \centering
        \includegraphics[width=0.4\linewidth]{general_figures/diplomacy} \\
        \cite{niculae-15,sayeed-12,boyd-graber-10}
    \end{block}




\end{columns}

}



\frame{

	\frametitle{Thanks}

        \begin{block}{Collaborators}
          \textsc{naqt}, Hal Daum\'e III (UMD), Anupam Guha (Maryland), Manjhunath Ravi (Colorado), Danny Bouman (UMD UG),
          Stephanie Hwa (UMD UG)
        \end{block}

	\begin{columns}
	
	\column{.5\linewidth}
        \begin{block}{Funders}
        \begin{center}
          \includegraphics[width=0.4\linewidth]{general_figures/nsf}
       \end{center}
        \end{block}
        
	\column{.5\linewidth}        
        \begin{block}{Supporters}
        	\gfxq{naqt}{.4}
        \end{block}

        \end{columns}
}




\begin{frame}{References}
\bibliographystyle{style/acl}
\tiny
\bibliography{bib/journal-full,bib/jbg}
\end{frame}




	

\begin{frame}
	\frametitle{Learning which Features are Useful}

	\begin{itemize}
		\item Use how humans these data as a prior for supervised maxent model~\cite{daume-04}
		\item Prior for label $a$ and feature $f$ is a function of the number of buzzes $b$ and tf-idf~\cite{salton-68}
\begin{equation}
  \left[ \vphantom{\frac{a}{b}}\alpha \alert<4>{\ind{ b(a,f) > 0}} + \beta \alert<3>{ b(a,f)} + \gamma
  \right] \alert<2>{\mbox{tf-idf}(a,f)} .
\label{eq:meanweight}
\end{equation}
		\begin{itemize}
			\item $\alpha$, $\beta$, and $\gamma = 0$: na\"ive zero prior
			\item $\alpha$ and $\beta = 0$: linear transformation of the mean
			\item $\alpha$ and $\gamma = 0$: number of buzzes times tf-idf value of the features
		\end{itemize}

	\end{itemize}

\end{frame}

\begin{frame}
	\frametitle{Using buzzes as a prior}

\begin{equation*}
  \left[ \vphantom{\frac{a}{b}}\alpha \ind{ b(a,f) > 0} + \beta b(a,f) + \gamma
  \right] \mbox{tf-idf}(a,f) .
\end{equation*}

\begin{center}
\begin{tabular}{cccccc}
Answers & Weighting & $\alpha$ & $\beta$ & $\gamma$ & Error\footnote{Buzz and tf-idf computed on training data; grid search on dev data; error on test data} \\
\hline
\multirow{5}{*}{100} & zero & - & - & - & 0.22 \\
& tf-idf & - & - & 8.3 & 0.08 \\
&  buzz-binary & 10.7 & - & - & {\bf 0.06} \\
&  buzz-linear & - &  1.1 & - & 0.10 \\
& buzz-tier & - & 1.6 & 0.5 & 0.07 \\
\hline
\end{tabular}
\end{center}
\end{frame}





\begin{frame}
	\begin{center}

\vspace{-.6cm}
\begin{figure}[tb]
\centering

\subfigure[Buzzes over all Questions]{
\includegraphics[width=0.6\linewidth]{qb/buzz_cloud}
\label{fig:buzz_cloud}
}

\subfigure[Wuthering Heights Question Text]{
\includegraphics[width=0.45\linewidth]{qb/wuthering_heights_question}
\label{fig:wh_question}
}
\subfigure[Buzzes on Wuthering Heights]{
\includegraphics[width=0.45\linewidth]{qb/wuthering_heights_buzz}
\label{fig:wh_buzz}
}
\end{figure}


	\end{center}

\end{frame}



\begin{frame}
	\frametitle{Accuracy vs. Speed}

	\begin{center}
	  \includegraphics[width=0.8\linewidth]{qb/accuracy_vs_speed}
	  \end{center}

\end{frame}

\begin{frame}{How we could translate a sentence}

\only<1>{\gfxs{example_3}{.9}}
\only<2>{\gfxs{example_4}{.9}}
\only<3>{\gfxs{example_5}{.9}}
\only<4>{\gfxs{example_6}{.9}}
\only<5>{\gfxs{example_7}{.9}}
\only<6>{\gfxs{example_8}{.9}}
\only<7>{\gfxs{example_9}{.9}}
\only<8>{\gfxs{example_10}{.9}}
\only<9>{\gfxs{example_11}{.9}}
\only<10>{\gfxs{example_12}{.9}}
\only<11>{\gfxs{example_13}{.9}}
\only<12>{\gfxs{example_14}{.9}}
\only<13>{\gfxs{example_15}{.9}}
\only<14>{\gfxs{example_16}{.9}}
\only<15>{\gfxs{example_17}{.9}}
\only<16>{\gfxs{example_18}{.9}}
\only<17>{\gfxs{example_19}{.9}}
\end{frame}




\end{document}



\documentclass[compress]{beamer}

%\usepackage{beamerthemesplit}
\usepackage{xmpmulti}

\usepackage{graphicx,float,wrapfig, bbm}
\usepackage{amsfonts, bbold, comment}
\usepackage{mdwlist}
\usepackage{subfigure}
\usepackage{colortbl}

\usepackage{multirow}

\pgfdeclareimage[width=\paperwidth]{mybackground}{colorado/boulder.pdf}

\newcommand{\e}[2]{\mathbb{E}_{#1}\left[ #2 \right] }
\newcommand{\ind}[1]{\mathbb{I}\left[ #1 \right] }
\newcommand{\ex}[1]{\mbox{exp}\left\{ #1\right\} }
\newcommand{\g}{\, | \,}
\newcommand{\citename}[1]{#1 }


\newcommand{\gfx}[2]{
\begin{center}
	\includegraphics[width=#2\linewidth]{labeling/#1}
\end{center}
}


\usetheme[bullet=circle,                     % Use circles instead of squares for bullets.
          titleline=true,                    % Show a line below the frame title.
          showdate=true,                     % show the date on the title page
          alternativetitlepage=true,         % Use the fancy title page.
          titlepagelogo=general_figures/culogo,              % Logo for the first page.
          % Logo for the header on first page.
          headerlogo=general_figures/boulder_cs,
          ]{UCBoulder}

\usecolortheme{ucdblack}
\title[Labeling Science]{Efficient, Interpretable Labeling for Large
  Scientific Document Collections}
\author{ Jordan Boyd-Graber}
\date{July 10, 2015}

\institute[Boulder] % (optional, but mostly needed)
{University of Colorado Boulder}

\AtBeginSection[] % "Beamer, do the following at the start of every section"
{ \begin{frame} \frametitle{Outline} % make a frame titled "Outline"
\tableofcontents[currentsection] % show TOC and highlight current section
\end{frame} }

\begin{document}

\frame{
\titlepage
\tiny
}

\begin{frame}{Setting: Big Data, Inconsistent Metadata}

\begin{columns}

\column{.5\linewidth}

\begin{itemize}
  \item Abstracts from many different agencies
  \item No cohesive organization scheme
  \item Untapped promise: looking at the text
\end{itemize}


\column{.5\linewidth}
\gfx{papers}{.8}
\end{columns}

\end{frame}

\begin{frame}{Problem: Unanticipated Questions}

\begin{columns}

\column{.5\linewidth}
\gfx{wind}{.8}

\column{.5\linewidth}

\begin{itemize}
  \item Existing schemas cannot answer new questions
  \item E.g.: What are we funding in renewables in New Jersey?
  \item Goal: Find these slices with minimum cost but provide a high
    quality answer
\end{itemize}

\end{columns}

\end{frame}

\section{Framework for Understanding Tradeoffs}

\begin{frame}{Cost vs. Quality}
  \only<1>{\gfx{quality_cost_1}{.7}}
  \only<2>{\gfx{quality_cost_2}{.7}}
  \only<3>{\gfx{quality_cost_3}{.7}}
  \only<4>{\gfx{quality_cost_4}{.7}}
\end{frame}

\begin{frame}{Understanding Tradeoffs}
  \only<1>{\gfx{frontier_1}{.8}}
  \only<2>{\gfx{frontier_2}{.8}}
  \only<3>{\gfx{frontier_3}{.8}}
  \only<4>{\gfx{frontier_4}{.8}}
\end{frame}

\section{Topic Models}

\begin{frame}{Examples of topics from NSF}

\begin{columns}

\column{.5\linewidth}

\begin{block}{}
ocean, ice, climate, sea, marine
\end{block}

\begin{block}{}
cell, protein, cells, molecular, biology
\end{block}

\begin{block}{}
data, software, tools, community, development
\end{block}

\begin{block}{}
language, learning, human, brain, visual
\end{block}

\pause

\column{.5\linewidth}
Problems
\begin{itemize}
 \item Could be granularity mismatch
  \item Topics not always interpretable or relevant
  \item Take it or leave it
\end{itemize}

\end{columns}

\end{frame}

\begin{frame}{Can we do better?}

\gfx{alto_qc}{.5}

\centering
\only<2>{ALTO: Active Learning from Topic Overviews}

\end{frame}


\section{ALTO}



\begin{frame}{Show users topic model output}
  \gfx{intro_screen}{.8}
\end{frame}


\begin{frame}{User labels some documents}
  \gfx{labeling_main}{.8}
\pause
\centering
Creates complete label set

\end{frame}

\begin{frame}{Algorithm learns pattern}
  \gfx{AL_asking}{.8}
\pause
\centering
But probably imperfect
\end{frame}


\begin{frame}{Active learning guides user to problem areas}
  \gfx{labeling_AL}{.8}
\pause
\centering
New labels, areas of confusion for algorithm

\end{frame}


\begin{frame}{Early Experiments}
  \only<1>{\gfx{purity_1}{.95}}
  \only<2>{\gfx{purity_2}{.95}}
  \only<3>{\gfx{purity_3}{.95}}
  \only<4->{\gfx{purity_4}{.95}}

\only<5>{
\centering
Caveat: dataset with gold labels
}
\end{frame}

\section{Unscientific Examples}

\begin{frame}{Five diverse projects}

What will go here eventually: five NSF projects which we
will label using our method, topic models, wiki labels, and others

\end{frame}

\begin{frame}{How they are labeled}

\end{frame}

\begin{frame}{Measuring Quality}

\end{frame}

\end{document}

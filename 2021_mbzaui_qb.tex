\documentclass[xcolor=dvipsnames,xcolor=table]{beamer}


\usetheme[
          showdate=true,                     % show the date on the title page
          alternativetitlepage=true,         % Use the fancy title page.
          titlepagelogo=general_figures/shell,              % Logo for the fir\
st page.
          ]{UMD}

%\usetheme{Rochester}

%\usepackage{beamerthemesplit}
\usepackage{xmpmulti}

\usepackage{booktabs}
\usepackage{graphicx,float,wrapfig, bbm}
\usepackage{amsfonts, bbold, comment}
\usepackage{mdwlist}
\usepackage{tikz}
\usepackage{subfigure}
\usepackage{colortbl}

\usepackage{multirow}

\usetikzlibrary{shapes.geometric}
\definecolor{xred}{HTML}{DB4437}
\definecolor{xyellow}{HTML}{F4B400}
\definecolor{xgreen}{HTML}{0F9D58}

\newcommand{\nq}{\abr{nq}}
\newcommand{\qb}{\abr{qb}}
\newcommand{\squad}{\abr{sq}\small{u}\abr{ad}}
\newcommand{\triviaqa}{\abr{t}\small{rivia}\abr{qa}}
\newcommand*{\tcircle}[1]{\tikz[anchor=base,baseline=-2.5pt] \node[circle,fill=#1,scale=0.9] (X) {};}
\newcommand*{\tsquare}[1]{\tikz[anchor=base,baseline=-2.5pt] \node[fill=#1,scale=1.2] (X) {};}
\newcommand*{\tdiamond}[1]{\tikz[anchor=base,baseline=-2.5pt] \node[diamond,fill=#1,scale=0.7] (X) {};}
\newcommand*{\ttriangle}[1]{\tikz[anchor=base,baseline=-1.5pt] \node[regular polygon,regular polygon sides=3,fill=#1,scale=0.6] (X) {};}



\newcommand{\fsi}[2]{
\begin{frame}[plain]
\vspace*{-1pt}
\makebox[\linewidth]{\includegraphics[width=\paperwidth]{#1}}
\begin{center}
#2
\end{center}
\end{frame}
}

\newcommand{\abr}[1]{\textsc{#1} }
\newcommand{\pos}[1]{{\texttt{#1}}}
\newcommand{\e}[2]{\mathbb{E}_{#1}\left[ #2 \right] }
\newcommand{\ind}[1]{\mathbb{I}\left[ #1 \right] }
\newcommand{\ex}[1]{\mbox{exp}\left\{ #1\right\} }
\newcommand{\g}{\, | \,}
\newcommand{\citename}[1]{#1 }

\newcommand{\gfxs}[2]{
\begin{center}
	\includegraphics[width=#2\linewidth]{simtrans/#1}
\end{center}
}

\newcommand{\gfxq}[2]{
\begin{center}
	\includegraphics[width=#2\linewidth]{qb/#1}
\end{center}
}



\newenvironment{variableblock}[2]{%
  \setbeamercolor{block body}{#2}
  \begin{block}{#1}}{\end{block}}


\newcommand{\goodbad}[2]{

\begin{columns}

  \column{.5\linewidth}

\begin{variableblock}{Good}{bg=PineGreen,fg=white}
  #1
\end{variableblock}


  \column{.5\linewidth}

\begin{variableblock}{Bad}{bg=BrickRed,fg=white}
  #2
\end{variableblock}


\end{columns}

}



%\usecolortheme{ucdblack}
\title[HCQA]{Cooperative and Competitive Machine Learning through Question Answering}
\author{ Jordan Boyd-Graber et al.}
\date{2021}

\institute[Maryland] % (optional, but mostly needed)
{University of Maryland}

\begin{document}

\frame{
\titlepage
\tiny
}

\fsi{qb/turing}{Turing Test: Definition of AI (Image from Wall Street
  International)}


\fsi{qb/starcraft}{Image: DeepMind}
\fsi{qb/DeepBlue}{Peter Morgan/Reuters}

\begin{frame}{How Machines Answer Computers Today and how to Improve}

  \begin{itemize}
  \item Good and bad of
      \begin{itemize}
      \item SQuAD
      \item Jeopardy!
      \item Natural Questions
      \end{itemize}
    \item What we can learn from human \abr{qa} (game shows and trivia nerds)
    \item A better \textsc{qa} task / dataset
      \item How to make it challenging for computers
      \item What a question answering gameshow might look like
      \end{itemize}
\end{frame}

%\fsi{general_figures/tng_poker}{}

\begin{frame}{SQuAD (Rajpurkar \& Jia et al. '16)}


  \only<1>{\gfxq{squad_ex}{.7}}
  \only<2>{\gfxq{squad_leader}{.8}}

  \only<3>{  \goodbad{Very scalable annotation
process that can cheaply generate
large numbers of questions per
article.}{Annotating questions directly
from the context passages strongly
skews the data distribution. The task
then becomes reverse engineering the
annotators, rather than language
understanding.}   }
  \only<3>{  \goodbad{The online leaderboard allows
easy benchmarking of systems and
motivates competition.}{Answers as spans reduces the
task to multiple choice, and doesn't
allow questions with answers latent in
the text.}   }
  \only<4>{ \goodbad{Computers can do well on this task, often described as ``machine reading''.}{Weissenborn et al. 2017 reveal much of these successes are shortcuts: look for a year when the question has ``when'', word overlaps with the question, etc.}  }
  \only<5>{  \goodbad{Human upperbound sets reasonable goal.}{Allows mischaracterization of what it means to ``read''.}   }

  \only<4>{\begin{block}{Along with Canada and the United Kingdom, what country generally doesn’t refer to universities as private schools?}
In the United Kingdom and several other Commonwealth countries including Australia and Canada,
    \end{block}

    }

\end{frame}

\fsi{qb/human_reading}{SQuAD: Ignore Knowledge}

\fsi{qb/jeopardy}{IBM Watson: QA Solved!}


\begin{frame}{But is Jeopardy! about Knowledge?}

  \begin{columns}
    \column{.25\linewidth}
    \gfxq{planet_money}{.75}
    \gfxq{jennings}{.7}
    \gfxq{kenny_malone}{.7}
    \column{.7\linewidth}

    From \href{file:///Users/jbg/repositories/jbg-talks/qb/jennings-buzzer.mp3}{Planet Money} \\

    \small

    {\bf JENNINGS:} The deal with the buzzer is this. The buzzer is
    not live until Alex finishes reading the question. And if you buzz
    in before your buzzer goes live, \alert<1>{you actually lock yourself out
    for a fraction of a second}. So the big mistake on the show is
    people who are all adrenalized and are buzzing too quickly, too
    eagerly.

    \pause

    {\bf MALONE:} OK. To some degree, "Jeopardy!" is kind of a video game, and a \alert<2>{crappy video game where it's, like, light goes on, press button} - that's it.

    \pause

    {\bf JENNINGS:} (Laughter) Yeah.

    {\bf MALONE:} Is that true?

    {\bf JENNINGS:} I do like to think of it as a \alert<3>{beautiful art} and not a really crappy video game.

  \end{columns}

\end{frame}


\begin{frame}{To the Jeopardy! pedants\dots}

  \begin{itemize}
  \item I know that ``questions'' are technically called answers, but
    easier for the rest of the talk to call them questions
  \item While you can buzz in again after a short lockout, it's the
    kiss of death if somebody else can buzz in then
  \item Yes, Ken Jennings and Brad Rutter are the best players to ever play Jeopardy!
    No, I'm not forgetting anybody.

    \gfxq{holzhauer}{.5}

  \end{itemize}

\end{frame}


%\fsi{qb/naqt_jbg_2}{Statistics}

% Add in some trivia / QB image


%\fsi{qb/squad}{SQuAD: Rajpurkar et al., 2016 (Reason)}

%\fsi{qb/triviaqa}{TriviaQA: Joshi et al., 2017 (Experts)}


\begin{frame}{Google's Natural Questions}
  \only<1>{\gfxq{natural_questions}{1.0}}
  \only<2>{\gfxq{abraham}{1.0}}

  Kwiatkowski et al., 2019
\end{frame}

\begin{frame}{How Natural Questions is Built}
  \begin{itemize}
  \item Take frequent questions from Google
  \item Filter out bad stuff
  \item Put it through a search engine (restricted to Wikipedia)
    \item Have multiple annotators look for an answer span (\textit{a la} SQuAD)
  \end{itemize}
\end{frame}

\begin{frame}{This is Great!}
  \begin{itemize}
  \item Real questions from real people
  \item \emph{Someone else} verifies that there is an answer
  \item Question independent of source: much more variation
    \item Relatively big dataset
  \end{itemize}
\end{frame}

\fsi{qb/hcqa_ambigqa}{}



\begin{frame}{}

  \begin{columns}
    \column{.4\linewidth}
        \includegraphics[width=0.8\linewidth]{general_figures/ben}
    \column{.6\linewidth}



        \begin{block}{ {\bf \href{http://umiacs.umd.edu/~jbg//docs/2020_acl_trivia.pdf}{What Question Answering can Learn from Trivia Nerds} }}
{\bf Jordan Boyd-Graber} and Benjamin B\"orschinger.  \emph{Association for Computational Linguistics}, 2020
        \end{block}

  \end{columns}
\end{frame}

\begin{frame}{Ambiguous Questions}
  \begin{small}
  \rowcolors{2}{gray!25}{white}
  \begin{tabular}{p{7cm}p{3cm}}
    \toprule
    Question & Gold Answer \\
    \hline
    \alert<2>{when was the last time michigan won the championship} & 1989 \\
    \alert<3>{what year did the us hockey team won the olympics} & 1960 and 1980 \\
    \alert<4>{which supreme court judge has surved in international court of justice} & Dalveer Bhandari \\
    \alert<5>{where does puerto rico's power come from} & Puerto Rico Electric Power Authority \\
    \bottomrule
  \end{tabular}
\end{small}

\begin{block}{Assumptions\dots}
  \only<2>{NCAA Division I Men's Football}
  \only<3>{Men's competition}
  \only<4>{Indian Supreme Court}
  \only<5>{Electric power}
  \only<6>{Bias in favor of men, English-speaking world.  Ambiguity is arbitrarily resolved by search engine result.}
  \end{block}
\end{frame}




\begin{frame}{Incorrect}

  \begin{block}{what is the formula for chromium(ii) sulfate}
    Chromium(II) sulfate refers to inorganic compounds with the chemical formula CrSO$_4 \cdot n$ H$_2$O.
  \end{block}

  Gold Answer: ---

\end{frame}

\fsi{qb/squad_2}{Test uncertainty (Rajpurkar et al., 2018)}

\begin{frame}
	\frametitle{This isn't new!}
	\begin{columns}

	\column{.5\linewidth}
	\begin{itemize}
		\item Game called ``quiz bowl''
		\item Two teams play each other
		\begin{itemize}
			\item Moderator reads a question
			\item When a team knows the answer, they signal (``buzz'' in)
			\item If right, they get points; otherwise, rest of the question is read to the other team
		\end{itemize}
		\item Hundreds of teams in the US alone
                \only<2>{\item Example \dots}
	\end{itemize}

	\column{.5\linewidth}
	\includegraphics{qb/quizbowl}

	\end{columns}

      \end{frame}


\begin{frame}[t]
	\frametitle{Sample Question}

        The Swiss-Italian architect Pietro Antonio Solari
        \only<2->{built several fortified towers in this city, which
          often vied for power with its northern rival Tver. A ruler
          of this city prevailed in the} \only<3->{Great Stand on the
          Ugra River. A prince from this city was nicknamed for
          winning a battle on the} \only<4->{Don river. Partly because
          a ruler of this city married} \only<5->{Sophia Palaiologina,
          the niece of the last Byzantine Emperor, this city styled
          itself the} \only<6->{``Third Rome'' after the fall of
          Constantinople. Another prince of this city stopped paying
          tribute to the} \only<7->{Mongols in 1476, ending the
          ``Tatar yoke.''} \only<8->{The Grand Duchy headquartered in
          this city came to an end in 1547 with the ascension of}
        \only<9->{ Ivan IV, who made it his capital. For 10 points,
          name this city where Ivan III renovated the
          Kremlin,} \only<10->{the capital of Russia.}\\
        \vspace{.5cm} \only<11->{ {\bf Moscow} (Moskva / Muscovy)}

\end{frame}



\begin{frame}{The problem of ambiguity and annotation error}
  \only<1>{\gfxq{error_and_difficulty_0}{1.0}}
  \only<2>{\gfxq{error_and_difficulty_1}{1.0}}
  \only<3>{\gfxq{error_and_difficulty_2}{1.0}}
  \only<4>{\gfxq{error_and_difficulty_3}{1.0}}

\end{frame}


\begin{frame}{How to correctly deal with ambiguity}

  \begin{block}{You should not assume a finite set of answers are enough!}
Different QA runs very seldom return exactly the same answer strings, and it is quite difficult to determine automatically whether the difference between a new string and a judged string is significant with respect to the correctness of the answer. (Voorhees, 2008)
\end{block}

\begin{itemize}
\item Low-level staffers can rule based on common sense
\item Larger issues require escalation
\item Annual tournaments have ``adjudication board''
\item Process maximizes fairness
\end{itemize}

\href{file:///Users/jbg/repositories/jbg-talks/qb/endoscope.mov}{Jeopardy Example}

\end{frame}

\fsi{qb/jeopardy_adjudication_1}{}
\fsi{qb/jeopardy_adjudication_2}{}
\fsi{qb/jeopardy_adjudication_3}{}
\fsi{qb/jeopardy_adjudication_4}{}
\fsi{qb/jeopardy_adjudication_0}{}


\fsi{qb/efficient_qa.jpg}{NeurIPS 2020: Efficient QA (https://go.umd.edu/2020eqa)}




\begin{frame}
	\frametitle{Question Structure Enables Discriminability}

	\begin{columns}
		\column{.5\linewidth}

		\includegraphics[width=1.0\linewidth]{qb/jeopardy}


		\column{.5\linewidth}
		\begin{itemize}
                        \item Watson must decide to answer {\bf once}, after
                          complete question
                        \item Quiz Bowl: decide after each word
                        \item Obscure clues at start, easy at end
                        \item ``Gold standard'' in trivia community
		\end{itemize}

	\end{columns}

\end{frame}


\begin{frame}{How to approach this problem \dots}

    \only<1>{
  \begin{columns}
    \column{.5\linewidth}
    \gfxq{guess}{0.8}
    \column{.5\linewidth}
    \gfxq{buzzer}{0.8}
  \end{columns}
}
\only<2>{
   \gfxq{guess}{0.5}
}
\end{frame}



\begin{frame}{}

  \begin{columns}
    \column{.4\linewidth}
    \begin{center}
        \includegraphics[width=0.8\linewidth]{general_figures/mohit}
        \end{center}
    \column{.6\linewidth}
        \begin{block}{ {\bf \href{http://cs.colorado.edu/~jbg//docs/2014_emnlp_qb_rnn.pdf}{A Neural Network for Factoid Question Answering over Paragraphs}}}
\underline{\href{http://cs.umd.edu/~miyyer/}{Mohit Iyyer}}, {\bf Jordan Boyd-Graber}, Leonardo Claudino, Richard Socher, and Hal {Daum\'{e} III}.  \emph{Empirical Methods in Natural Language Processing}, 2014
        \end{block}

        \begin{block}{ {\bf \href{file:///Users/jbg/public_html/docs/2015_acl_dan.pdf}{Deep Unordered Composition Rivals Syntactic Methods for Text Classification}}}
\underline{\href{http://cs.umd.edu/~miyyer/}{Mohit Iyyer}}, Varun
Manjunatha, {\bf Jordan Boyd-Graber} and Hal {Daum\'{e} III}.  \emph{Empirical Methods in Natural Language Processing}, 2014
        \end{block}

  \end{columns}
\end{frame}
\begin{frame}{Vector Space Model}

  \only<1>{\gfxq{unigram_models_0}{.8}}
  \only<2>{\gfxq{unigram_models_1}{.8}}
  \only<3>{\gfxq{unigram_models_2}{.8}}
  \only<4>{\gfxq{unigram_models_3}{.8}}
  \only<5>{\gfxq{unigram_models_4}{.8}}
  \only<6>{\gfxq{unigram_models_5}{.8}}
  \only<7>{\gfxq{unigram_models_6}{.8}}
  \only<8>{\gfxq{unigram_models_7}{.8}}
  \only<9>{\gfxq{unigram_models_8}{.8}}

\end{frame}


\begin{frame}{Non-linear improvement: Deep Averaging Networks}

  \only<1>{\gfxq{dan_1}{.8}}
  \only<2>{\gfxq{dan_2}{.6}}
  \only<3>{\gfxq{dan_3}{.6}}
  \only<4>{\gfxq{dan_4}{.6}}

\end{frame}



% \begin{frame}{Training}

%   \begin{columns}
%     \column{.5\linewidth}
%       \begin{itemize}
%         \item Initialize embeddings from \textsc{word2vec}
%         \item Randomly initialize composition matrices
%         \item Update using \textsc{warp}
%           \begin{itemize}
%             \item Randomly choose an instance
%             \only<2->{\item Look where it lands}
%             \only<4->{\item Has a correct answer}
%             \only<5->{\item Wrong answers may be closer}
%             \only<6->{\item Push away wrong answers
%             \item Bring correct answers closer}
%           \end{itemize}
%       \end{itemize}

%     \column{.5\linewidth}

%       \only<1>{\gfxq{warp_training_5}{.8}}
%       \only<2>{\gfxq{warp_training_4}{.8}}
%       \only<3>{\gfxq{warp_training_3}{.8}}
%       \only<4>{\gfxq{warp_training_2}{.8}}
%       \only<5>{\gfxq{warp_training_1}{.8}}
%       \only<6>{\gfxq{warp_training_0}{.8}}
%   \end{columns}

% \end{frame}


\begin{frame}{More complicated representations}

  \gfxq{embedding}{1.0}

\end{frame}

\fsi{qb/delft_intuition}{\underline{\href{http://users.umiacs.umd.edu/~chenz/}{Chen Zhao}}, Chenyan Xiong, Xin Qian, and {\bf Jordan Boyd-Graber}.  {\bf \href{http://umiacs.umd.edu/~jbg//docs/2020_www_delft.pdf}{Complex Factoid Question Answering with a Free-Text Knowledge Graph}}.  \emph{The Web Conference}, 2020.}

\begin{frame}{What Mohit did afterward\dots}

  \only<1>{\gfxq{elmo_idea}{.9}}
  \only<2>{\gfxq{elmo_paper}{.9}}


\end{frame}

\begin{frame}{How to approach this problem \dots}

    \only<1>{
  \begin{columns}
    \column{.5\linewidth}
    \gfxq{guess}{0.8}
    \column{.5\linewidth}
    \gfxq{buzzer}{0.8}
  \end{columns}
}
\only<2>{
   \gfxq{buzzer}{0.5}
}
\end{frame}


\begin{frame}{}

  \begin{columns}
    \column{.5\linewidth}
        \includegraphics[width=0.7\linewidth]{general_figures/hehe}
    \column{.5\linewidth}
        \begin{block}{{\bf
              \href{http://cs.colorado.edu/~jbg//docs/qb_emnlp_2012.pdf}{Besting
                the Quiz Master: Crowdsourcing Incremental
                Classification Games}}}

          {\bf Jordan Boyd-Graber}, He He, and Hal {Daum\'{e} III}. \emph{Empirical Methods in Natural Language Processing}, 2012
        \end{block}

        \begin{block}{{\bf
              \href{http://www.cs.colorado.edu/~jbg/docs/2016_icml_opponent.pdf}{Opponent Modeling in Deep Reinforcement Learning}}}

          He He, {\bf Jordan Boyd-Graber}, Kevin Kwok, and Hal
          {Daum\'{e} III}. \emph{International Conference of Machine Learning}, 2016
        \end{block}

  \end{columns}
\end{frame}


\begin{frame}
\frametitle{Interface}

\begin{columns}

	\column{0.5\linewidth}

	\begin{center}
		\includegraphics[width=0.8\linewidth]{qb/screenshot}
	\end{center}

	\column{0.5\linewidth}


	\only<2>{
	\begin{itemize}
		\item 7000 questions: first day
		\item 43000 questions: two weeks
		\item 461 unique users
                \item Imitated \dots
	\end{itemize}
        \gfxq{protobowl}{.8}
	}



\end{columns}
\end{frame}

\begin{frame}{Not all opponents are equal}

  \gfxq{player_profile}{.9}

  \pause
  Varies by category!

\end{frame}




\begin{frame}{Many models, choose which: DRON-MoE}

  \only<1>{\gfxq{dron-moe}{.8}}
  \only<2>{\gfxq{dron-moe2}{.8}}

\end{frame}

\begin{frame}{Reward}

  \gfxq{dqn_results}{.5}

\end{frame}

\begin{frame}{Reward: Closer Look}

  \only<1>{\gfxq{reward1}{.8}}
  \only<2>{\gfxq{reward2}{.8}}
  \only<3>{\gfxq{reward3}{.8}}
  \only<4>{\gfxq{reward4}{.8}}
\end{frame}

\begin{frame}{Experiment 1}

		\begin{columns}
			\column{.25\linewidth}
				\gfxq{colby_jeo}{1.0}
                                Colby Burnett:
                                \$375,000
			\column{.25\linewidth}
				\gfxq{ben_jeo}{1.0}
                                Ben Ingram:
                                \$427,534
			\column{.25\linewidth}
				\gfxq{alex_jeo}{1.0}
                                Alex Jacobs: \$151,802
			\column{.25\linewidth}
				\gfxq{kristin_jeo}{1.0}
                                Kristin Sausville: \$95,201
		\end{columns}

                \pause


                \begin{center}
                End result: 200-200 tie!
                \end{center}

\end{frame}

\fsi{qb/hsnct1}{}
\fsi{qb/jennings_handshake}{300-160}
% \fsi{qb/hsnct_2016}{330-60} % Not sure of exact score, double check
\fsi{qb/nasat}{Humans 345-145 (vs. OUSIA)}

\fsi{qb/hsnct_2017}{Best Human Trivia Players {\bf Barely} Defeated,
  June 2017}



\fsi{qb/hcqa}{Best Human Trivia Players {\bf Soundly} Defeated,
  December 2017 \\ (OUSIA)}



\fsi{qb/seattle_crowd}{Question Answering is Not a Trivial Activity}
\fsi{qb/chicago_crowd}{http://qanta.org}

\fsi{qb/boring_dot_products}{Boring Dot Products}

\fsi{general_figures/blackbox}{}

\fsi{interpretability/lime_explanation}{LIME: Ribeiro et al. (2016)}

\fsi{interpretability/lime_image_explain}{LIME: Ribeiro et al. (2016)}

\fsi{interpretability/mt_task}{LIME: Ribeiro et al. (2016)}

\fsi{interpretability/mt_results}{What can humans do for ML?}

\fsi{simtrans/centaur-chess}{Centaur Chess}


\begin{frame}{}

  \begin{columns}
    \column{.4\linewidth}
    \begin{center}
        \includegraphics[width=0.8\linewidth]{general_figures/shi}
        \end{center}
    \column{.6\linewidth}
        \begin{block}{{\bf What can AI do for me: Evaluating Machine Learning Interpretations in Cooperative Play}} \underline{\href{http://users.umiacs.umd.edu/~shifeng/}{Shi Feng}} and {\bf Jordan Boyd-Graber}. \emph{Intelligent User Interfaces}, 2019
        \end{block}

  \end{columns}
\end{frame}



\begin{frame}{Team-Based Interpretability}

  \only<1>{\gfxq{qb_centaur_1}{.9}}
  \only<2>{\gfxq{qb_centaur_2}{.9}}
  \only<3>{\gfxq{qb_centaur_3}{.9}}
  \only<4>{\gfxq{qb_centaur_6}{.9}}

\end{frame}


\fsi{qb/augment/screenshot_all}{Interface}

\fsi{qb/augment/screenshot_guesses}{}

\fsi{qb/augment/screenshot_highlight}{{\bf Highlighting}}

\fsi{qb/augment/screenshot_evidence}{}

\begin{frame}{Experts vs. Novices}

 \begin{block}{Experts}
   Trivia experts, familiar with task, enjoy the task
 \end{block}

 \begin{block}{Mechanical Turkers}
   Mechanical Turkers: easily overwhelmed, need the help
 \end{block}

\end{frame}

\fsi{qb/augment/tools_acc}{Evidence helps novices, experts are expert}
\fsi{qb/augment/tools_buzz}{Hights help experts}

\begin{frame}{Regression Analysis}
    For each triple (player, question, interpretations), we predict the outcome
    (correct answer or not) with a logistic regression. The features include:
    \begin{itemize}
        \item player ID
        \item question ID
        \item buzzing position
        \item enabled interpretations: individual and combinations
    \end{itemize}

    \pause

    \begin{block}{Coefficients tell story!}
      \begin{itemize}
        \item {\bf Big, Positive}: Help
        \item {\bf Big, Negative}: Hurt
        \item {\bf Small}: Neutral
      \end{itemize}
    \end{block}

\end{frame}


\fsi{qb/augment/coefs_0}{Everything helps: Evidence for novies,
  Highlight for experts}
\fsi{qb/augment/coefs_1}{Synergistic effects}
\fsi{qb/augment/coefs_2}{Highlight and evidence help experts most}
\fsi{qb/augment/coefs_3}{For novices, less synergy}

\begin{frame}{Improvement through Reinforcement Learning}

  \only<1>{\gfxq{rl_centaur_2}{.9}}
  \only<2>{\gfxq{rl_centaur_3}{.9}}
  \only<3>{\gfxq{rl_centaur_4}{.9}}
  \only<4>{\gfxq{rl_centaur_5}{.9}}
  \only<5>{\gfxq{rl_centaur_6}{.9}}

\end{frame}



\begin{frame}{}

  \begin{columns}
    \column{.4\linewidth}
    \begin{center}
        \includegraphics[width=0.8\linewidth]{general_figures/shi}
        \end{center}
    \column{.6\linewidth}
    \begin{block}{\href{http://umiacs.umd.edu/~jbg//docs/2018_emnlp_rs.pdf}{Pathologies of Neural Models Make Interpretation Difficult}}
      \underline{\href{http://users.umiacs.umd.edu/~shifeng/}{Shi Feng}}, \underline{\href{http://www.ericswallace.com/}{Eric Wallace}}, Alvin Grissom II, \underline{\href{https://www.entilzha.io/}{Pedro Rodriguez}}, Mohit Iyyer, and {\bf Jordan Boyd-Graber}.  \emph{Empirical Methods in Natural Language Processing}, 2018
        \end{block}

  \end{columns}
\end{frame}

\begin{frame}{How to highlight words?}

To interpret a model prediction on an input sequence of $n$
words~$\vec{w}=\langle\vec{w}_1, \vec{w}_2, \ldots
\vec{w}_n\rangle$, we approximate the classifier $f$ with a linear
function of $w_i$ derived from the first-order Taylor expansion. The
importance of $w_i$, with embedding $\vec{v}_i$, is the derivative
of $f$ with respect to the one-hot vector:
\begin{equation} \frac{\partial f}{\partial w_i} \
   = \frac{\partial f}{\partial \vec{v}_i}\frac{\partial \vec{v}_i}{\partial w_i} \
   = \frac{\partial f}{\partial \vec{v}_i} \cdot \vec{v}_i.
\end{equation}
This simulates how model predictions change when a particular word's embedding is set to the zero vector---it approximates word removal~\cite{ebrahimi2017hotflip,wallace2018Neighbors}.

\end{frame}


\begin{frame}{Neural Models are Brittle}

  \gfxq{pathologies_what_company}{.8}

\end{frame}


\begin{frame}{Can we improve QA systems?}

\begin{columns}
  \column{.6\linewidth}
     \gfxq{trick/pyramid}{.9}
     \column{.4\linewidth}
     \begin{itemize}
       \item Questions should be pyramidal
       \item But for whom?
         \begin{itemize}
           \item Quotes
           \item Reusing clues
         \end{itemize}
         \item Adversarial writing
         \item Improve questions
     \end{itemize}
\end{columns}
\end{frame}


\begin{frame}{Adversarial Examples}

  \gfxq{turtle_rifle}{0.9}

  Athalye, Engstrom, Ilyas, and \alert<2>{Kwok}.  Synthesizing Robust Adversarial
  Examples.  \emph{ICML}, 2018.

\end{frame}

\begin{frame}{What do we mean by ``adversarial''?}

  \gfxq{trick/flow_chart_horizontal_label}{1.0}

  \begin{itemize}
    \item Round 1: Only IR interpretations
    \item Round 2: IR and RNN (influence functions) interpretations
      \pause
    \item Another reason we need to have good explanations of QA
  \end{itemize}

\end{frame}

\fsi{qb/trick/brahms_0}{\href{http://write.qanta.org}{http://write.qanta.org}}
\fsi{qb/trick/brahms_1}{}
\fsi{qb/trick/brahms_2}{}
\fsi{qb/trick/brahms_3}{}
\fsi{qb/trick/brahms_4}{}
\fsi{qb/trick/brahms_5}{}


\fsi{qb/trick/round_one}{Round 1: Only IR-based QA system}
\fsi{qb/trick/round_two}{Round 2: RNN-based QA system}


\begin{frame}{Competition}

  \gfxq{trick/pace}{.8}

\begin{itemize}
  \item December 15: Seven top human teams, fourteen computer teams
  \item Top four teams from each ``division'' faced off against each
    other
    \pause
  \item All computer teams lost to human teams
    \pause
  \item But two games were really close; strongest system was based on BERT
  \item \href{http://events.qanta.org}{http://events.qanta.org}
\end{itemize}

\end{frame}

% Hard but not really a good question: F3 Q24



\begin{frame}{Matching Entites Across Sentences}

\begin{block}{\only<2->{Magic Flute}}

    At its premiere, \alert<3>{the librettist of this opera} portrayed
    \alert<4>{a character who asks for a glass of wine with his dying wish}. \alert<4>{That
    character} in this opera is instructed to ring some bells to summon
    his love. At its beginning, \alert<5>{a man} who claims to have killed a (*)
    serpent has a padlock put on \alert<5>{his} mouth because of \alert<5>{his} lying. The
    plot of this opera concerns a series of tests that \alert<5>{Tamino} must
    undergo to rescue Tamina from Sorastro. For 10 points, name this
    Wolfgang Mozart opera titled for \alert<6>{an enchanted woodwind instrument}.
\end{block}



\only<3-4>{{\bf Not all references are named (\alert<3>{Emanuel
      Schikaneder}, \alert<4>{Papageno})}}
\only<5>{Need to be able to match pronouns across sentences (or have
  deep world knowledge)}
\only<6>{Requires semantic knowledge}
\end{frame}


\begin{frame}{Linguistics FTW}

  The main character of a story by \alert<2>{this author opens Crime and Punishment} to a
random page, but finds it to be a copy of The Brother Karamazov, and equates
himself with Monsieur Bovary. This author wrote a story in which the priest
Naigu undergoes a boiling treatment to decrease the size of his nose. This
author of "Cogwheels" wrote about two people who steal to survive near the
southern gate of Kyoto in a story that features inconsistent accounts from a
woodcutter, a priest, a widow, and the ghost of a samurai. For 10 points, name
this author of "Rashomon" and namesake of a Japanese literary prize. \\
\only<3->{\textbf{Answer}: Ryunosuke Akutagawa}
\end{frame}


\begin{frame}{}

  \begin{columns}
    \column{.4\linewidth}
        \includegraphics[width=0.8\linewidth]{general_figures/hehe} \\
        \includegraphics[width=0.8\linewidth]{general_figures/alvin}
    \column{.6\linewidth}

        \begin{block}{ {\bf \href{http://umiacs.umd.edu/~jbg//docs/2015_emnlp_rewrite.pdf}{Syntax-based Rewriting for Simultaneous Machine Translation}}}
He He, Alvin Grissom II, {\bf Jordan Boyd-Graber}, and Hal {Daum\'{e} III}.  \emph{Empirical Methods in Natural Language Processing}, 2015
        \end{block}

        \begin{block}{ {\bf \href{http://umiacs.umd.edu/~jbg/docs/2016_naacl_interpretese.pdf}{Interpretese vs. Translationese: The Uniqueness of Human Strategies in Simultaneous Interpretation}}}
He He, {\bf Jordan Boyd-Graber}, and Hal {Daum\'{e} III}.
\emph{North American Association for Computational Linguistics}, 2016
        \end{block}

  \end{columns}


\end{frame}

\begin{frame}{Simultaneous Interpretation is Hard!}

  \begin{columns}
    \column{.5\linewidth}
  \begin{itemize}
    \item Exhausting for humans
    \item Computers not trusted
    \item Differential strengths
    \item Same word-by-word characteristic
  \end{itemize}

  \column{.5\linewidth}
 \gfxs{computer-interpreter}{1.0}
 \end{columns}
\end{frame}

\begin{frame}{How we could translate a sentence}

\only<1>{\gfxs{example_3}{.9}}
\only<2>{\gfxs{example_4}{.9}}
\only<3>{\gfxs{example_5}{.9}}
\only<4>{\gfxs{example_6}{.9}}
\only<5>{\gfxs{example_7}{.9}}
\only<6>{\gfxs{example_8}{.9}}
\only<7>{\gfxs{example_9}{.9}}
\only<8>{\gfxs{example_10}{.9}}
\only<9>{\gfxs{example_11}{.9}}
\only<10>{\gfxs{example_12}{.9}}
\only<11>{\gfxs{example_13}{.9}}
\only<12>{\gfxs{example_14}{.9}}
\only<13>{\gfxs{example_15}{.9}}
\only<14>{\gfxs{example_16}{.9}}
\only<15>{\gfxs{example_17}{.9}}
\only<16>{\gfxs{example_18}{.9}}
\only<17>{\gfxs{example_19}{.9}}
\end{frame}

\fsi{simtrans/interpreter-screenshot}{Same human-computer cooperation}

\begin{frame}{Ongoing Work: Modeling Questions and Systems}
  \gfxq{leaderboard_model}{.7}

  \begin{footnotesize}
  \begin{enumerate*}
  \item For each subject $j$:
  \begin{enumerate*}
    \item Draw skill $\theta_j \sim \mathcal{N}(\mu_\theta, \tau_\theta^{-1})$
  \end{enumerate*}
  \item For each item $i$:
  \begin{enumerate*}
    \item Draw difficulty $\beta_i \sim\mathcal{N}(\mu_\beta, \tau_\beta^{-1})$
    \item Draw discriminability $\gamma_i \sim \mathcal{N}(\mu_\gamma, \tau_\gamma^{-1})$
    \item Draw $\lambda_i \sim \text{U}[0,1]$
  \end{enumerate*}
  \item Draw subject $i$ response on item $j$, $r_{ij} \sim p_{ij}(r_{ij} \mid \theta_j, \beta_i, \lambda_i )=$
    \begin{equation}
      p_{ij}(r_{ij}=1|\theta_j)=\frac{\lambda_i}{1+e^{-\gamma_i(\theta_j-\beta_i)}}.
      \label{eq:isicle:ours}
    \end{equation}
  \end{enumerate*}
  \end{footnotesize}
\end{frame}

\begin{frame}{Improving Leaderboards}
  \only<1>{\gfxq{leaderboard_fit}{.75}}
  \only<2>{\gfxq{leaderboard_active}{.8}}
\end{frame}

\begin{frame}{Educational Applications}
  \begin{center}
    http://karl.qanta.org
  \end{center}
  \gfxq{karl}{0.9}

\end{frame}


\begin{frame}{Improving Representation}
        \begin{tabular}{c l rr rr rr rr}
        % \hline
        % \multicolumn{2}{c}{\small{\textbf{Demography}}}  & \textbf{\nq} & \textbf{\qb} & \textbf{\squad} & \textbf{\triviaqa} \\
        \hspace{15pt} & \multirow{2}{*}{\textbf{Value}} %{\small{\textbf{Value}}}
            & \multicolumn{2}{c}{\textbf{\nq}}
            & \multicolumn{2}{c}{\textbf{\qb}}
            & \multicolumn{2}{c}{\textbf{\squad}}
            & \multicolumn{2}{c}{\textbf{\triviaqa}} \\

        &
            & \hspace{5pt}\textbf{Train} & \textbf{Dev}
            & \hspace{5pt}\textbf{Train} & \textbf{Dev}
            & \hspace{5pt}\textbf{Train} & \textbf{Dev}
            & \hspace{5pt}\textbf{Train} & \textbf{Dev}\\

        \toprule
        \multirow{3}{*}{\rotatebox[origin=c]{90}{\textbf{Gender}}}
            & Male
                &  \textbf{75.67} &  \textbf{76.33}
                &  \textbf{91.77} &  \textbf{91.63}
                &  \textbf{87.82} &  \textbf{95.15}
                &  \textbf{83.76} &  \textbf{83.32} \\

            & Female
                &  27.47 &  27.56
                &  10.29 &   9.87
                &  13.44 &   5.20
                &  20.54 &  20.29 \\

                % &   0.31 &   0.35 &   0.27 &   0.30
            % & Other           &   0.12 &   0.00 &   0.09 &   0.12 \\
            & No Gender
                &   0.31  &   0.47
        		&   0.35  &   0.39
        		&   0.27  &   0.00
        		&   0.30  &   0.35 \\
        \midrule
        \multirow{3}{*}{\rotatebox[origin=c]{90}{\textbf{Country}}}
            & US
                & \textbf{59.62}	& \textbf{58.66}
        		& \textbf{29.70}	& \textbf{26.28}
        		& \textbf{32.74}	& \textbf{24.93}
        		& 31.32	& 30.91
                \\

            & UK
        		& 15.76	& 15.78
        		& 17.92	& 17.68
        		& 19.66	& 16.83
        		& \textbf{41.92}	& \textbf{41.32}
        		\\

            & France
        		& 1.79	& 1.18
        		& 10.06	& 10.34
        		& 7.76	& 10.57
        		& 4.37	& 4.84
        		\\


        \midrule
        \multirow{4}{*}{\rotatebox[origin=c]{90}{\textbf{Field}}}
            & Film/TV
				& \textbf{39.19}	& \textbf{37.93}
				& 3.16	& 1.89
				& 10.72	& 1.32
				& 20.64	& \textbf{20.75}
				\\

            & Writing
				& 7.40	& 6.95
				& \textbf{36.62}	& \textbf{36.39}
				& 10.46	& 6.70
				& 18.41	& 18.05
				\\

            & Politics
				& 11.98	& 10.84
				& 24.02	& 24.86
				& \textbf{36.97}	& \textbf{46.61}
				& \textbf{21.18}	& 20.72
				\\

            & Science/Tech
				& 3.61	& 4.71
				& 8.93	& 7.50
				& 13.67	& 29.60
				& 5.43	& 5.54
				\\
\bottomrule \end{tabular}
\end{frame}



\begin{frame}{Future \dots}

  \begin{itemize}
    \item Computers dominate on ``normal'' questions
    \item Not so much on adversarial questions
    \item Need to expand explanations
      \begin{itemize}
        \item Measured on real-world tasks
        \item With more diverse users
        \item Increase the breadth of resources
        \end{itemize}
      \item Multilingual and multicultural
  \end{itemize}

\end{frame}





\begin{frame}{Find out More!}

		\begin{itemize}
			\item Code: \url{http://github.com/Pinafore/qb}
                        \item Shared Task \url{http://qanta.org}
		\end{itemize}

\end{frame}

% \begin{frame}{Come to UMD}

% \begin{columns}
% 	\column{.5\linewidth}
%         \only<1>{
%         	\begin{center}
% 		\includegraphics[width=.9\linewidth]{umd/umd} \\
% 		\includegraphics[width=.9\linewidth]{umd/qb_team}
% 	\end{center}
%               }
% 	\column{.5\linewidth}
% 		\begin{itemize}
%                 \item Looking for undergrads/grads/interns
%                 \item A great place for natural language
%                   processing and machine learning
%                 \item Not too shabby at quiz bowl either
% 		\end{itemize}
% \end{columns}

% \end{frame}

\frame{
  \frametitle{But wait, there's more!}

  \vspace{-.5cm}

\begin{columns}



  \column{.5\linewidth}

   \begin{block}{Computational Social Science}
     \centering
     \includegraphics[width=0.9\linewidth]{teaparty/figures/framing} \\
     \cite{nguyen-13b,nguyen-15}
   \end{block}


    \begin{block}{Interactive Machine Learning}
     \centering
        \includegraphics[width=0.4\linewidth]{interactive_topic_models/new_interface} \\
       \cite{Poursabzi-16, Yuan-20:cold,Yuan-20:refinement}
    \end{block}


  \column{.5\linewidth}


    \begin{block}{Multilingual Topic Models}
      \begin{center}
        \begin{large}
          $p_{\mbox{topic}}(e | f)$ \\
         \end{large}
      \cite{eidelman-12,hu-14,Yuan-18}
       \end{center}
    \vspace{-.3cm}
    \end{block}


    \begin{block}{Sentiment / Internal State}
    \centering
        \includegraphics[width=0.4\linewidth]{general_figures/diplomacy} \\
        \cite{niculae-15,peskov-20}
    \end{block}




\end{columns}

}




\frame{

	\frametitle{Thanks}

        \begin{block}{Collaborators (not listed on previous slides)}
          Anupam Guha (Maryland), Manjhunath Ravi (Colorado), Danny Bouman (UMD UG),
          Stephanie Hwa (UMD UG), Yogarshi Vyas (UMD), Larry Davis
          (UMD), Naho Orita (Tohoku), Snigdha Chaturvedi (UMD), Varun
          Manjunatha (UMD), Srijan Kumar (UMD), Vlad Niculae
          (Cornell), Cristian Danescu-Niculescu-Mizil (Cornell),
          Richard Socher (Salesforce), Leonardo Claudino (UMD)
        \end{block}

	\begin{columns}

	\column{.5\linewidth}
        \begin{block}{Funders}
        \begin{center}
          \includegraphics[width=0.4\linewidth]{general_figures/nsf}
       \end{center}
        \end{block}

	\column{.5\linewidth}
        \begin{block}{Supporters}
        	\gfxq{naqt}{.4}
        \end{block}

        \end{columns}
}

\begin{frame}[plain]

\begin{columns}
  \column{.3\linewidth}
        \begin{center}
          @boydgraber
          \includegraphics[width=0.6\linewidth]{general_figures/twitter}
          \\
          \end{center}
  \column{.65\linewidth}

  \begin{center}
    https://www.youtube.com/c/JordanBoydGraber
    \includegraphics[width=1.0\linewidth]{general_figures/youtube} \\

\end{center}

\end{columns}

\begin{center}
\huge
http://qanta.org \\
http://boydgraber.org
       \end{center}


\end{frame}

\begin{frame}{References}
\bibliographystyle{style/acl}
\tiny
\bibliography{bib/journal-full,bib/jbg,bib/pedro}
\end{frame}




\begin{frame}{How can this fail?}

  \only<1>{\gfxq{opponent_fail1}{.8}}
  \only<2>{\gfxq{opponent_fail2}{.8}}
  \only<3>{\gfxq{opponent_fail3}{.8}}
  \only<4>{\gfxq{opponent_fail4}{.8}}
  \only<5>{\gfxq{opponent_fail5}{.8}}
  \only<6>{\gfxq{opponent_fail6}{.8}}
  \only<7>{\gfxq{opponent_fail7}{.8}}

\end{frame}

\begin{frame}{Can we do better?}

  \only<1>{\gfxq{dqn_overview2}{.8}}
  \only<2>{\gfxq{dqn_overview3}{.8}}
  \only<3>{\gfxq{dqn_overview4}{.8}}

\end{frame}



\begin{frame}{Impossible Until the End}
\alert<3>{Ritchie Watson commended this play's historical accuracy for
  getting the price for a dozen eggs right---ten cents---to defend
  against Elizabeth Hardwick’s contention that it was a sentimental
  history.} \alert<4>{At the end of this play, a man wonders why a wheelchair is
at the top of a staircase, and} \alert<5>{Alexandra announces that she is leaving
her mother. Leo is pressured into stealing a set of valuable railroad
bonds in this drama. In this play, which takes its title from the Song
of Solomon,} Regina Hubbard schemes  \tdiamond{xgreen}
\tcircle{xgreen} to obtain a majority share in a cotton mill. For 10
points,  \tsquare{xgreen}  \ttriangle{xgreen} name this play by
Lillian Hellman. \\

\pause

\textbf{Answer}: The\ Little\ Foxes\\

\only<3>{Academic literature}
\only<4>{Vague plot summary}
\only<5>{Avoid last names}

\end{frame}

% Tricky and hard: F3 Question 25

\begin{frame}{Tricky and impossible for current systems}
In Our Town, a character  \ttriangle{xred} with this given name explains Grover’s Corners’ place in the universe. In The Crucible, a character  \tcircle{xred} with this first name contends that the girls’ actions are part of their “silly seasons” and is the wife of Francis Nurse. A novel with this name, which conducts hidden messages to Rommel in The English Patient, is titled for a character who is killed in a boating accident at Manderley. For 10 points, give this name of a Daphne du Maurier  \tdiamond{xyellow}  \tsquare{xgreen} gothic novel which is also the first name of Miss Sharp, the protagonist of William Thackeray's Vanity Fair. \tcircle{gray}  \ttriangle{gray}


\ttriangle{xred} Thornton Wilder
\tcircle{xred} Richard
\tcircle{gray} Richard
\ttriangle{gray} Thornton Wilder \\
\textbf{Answer}: Rebecca\\
\end{frame}

% Tricky and hard: F3 Question 29

\begin{frame}{Close, but \dots}
An army that took its name from this geographical feature had a
British doctor, James Paroissien, as its Surgeon General, and recent
scholarship by Peter Blanchard revealed its use of slaves. That army
crossed this geographical feature according  \ttriangle{xred} to
Thomas Maitland’s  \tcircle{xred} plans at Uspallata and Los
Patos. Spanish forces under Rafael Maroto were defeated in the
foothills of these mountains by an army led by José de San Martín and
Bernardo O'Higgins.  \tsquare{xred} For 10 points, name this mountain
range of South America that played a role  \ttriangle{xyellow} in the
independence of Chile. \tdiamond{gray}  \tcircle{gray}  \tsquare{gray}

\ttriangle{xred} Angel Falls
\tcircle{xred} Mountain
\tsquare{xred} Battle of Chacabuco
\tdiamond{gray} Battle of Chacabuco
\tcircle{gray} Mountain
\tsquare{gray} Battle of Chacabuco \\
\pause
\textbf{Answer}: Andes\\

\end{frame}

% Showing strength of QB format: Packet F3 Question 30

\begin{frame}{Showing the strength of quiz bowl format}
A painting by this artist  \tsquare{xred} presents a tree as a cupboard with two doors, behind which are a white sphere and a miniature lighted house. A series of three paintings by this man shows a daytime sky with cumulus clouds above a dark house illuminated by a lamppost. This artist of Blood Will Tell and The Empire of Lights depicted a clock set to 12:43  \tcircle{xgreen} as a locomotive speeds out of a fireplace and a man in  \tsquare{xyellow} a bowler hat with an  \ttriangle{xgreen} apple in front of his face. For 10 points, name this Belgian surrealist painter  \tdiamond{xgreen} of the Son of Man and Time Transfixed.

\tsquare{xred} Argon \\
\pause
\textbf{Answer}: René\ Magritte\\
\end{frame}

% Why it's a good learning framework: Packet F3 Question 31


\begin{frame}{Our QA Systems are Crazy}

\only<1-2>{
\begin{block}{Out of Touch}
On "The Critic", this man was replaced by a drunken animatronic bear from the Country Bear Jamboree but nobody seemed to notice.  This man was edited to appear in the film "Contact", which prompted an angry statement from Mike McCurry.  He was portrayed by Dennis Quaid in "The Special Relationship", who ate McDonald's every day to prepare for the role.  A fictionalized version of this man named Henry Burton, a charismatic Southern governor running for the Democratic nomination, is portrayed by John Travolta in "Primary Colors".  For ten points, name this American president who played the saxophone on an appearance on the Arsenio Hall Show.
\end{block} }
\only<2>{{\bf Samuel L. Jackson?}}

\end{frame}

\begin{frame}{Comparing Models}

  \begin{itemize}
    \item Single-Player
    \item Deep Q-Network (DQN): World=Opponent~\cite{mnih-15}
      \begin{itemize}
        \item Learn representation of state to estimate $Q$-function
        \item Generalization of regression-based methods
        \item Similar to our representation of content
      \end{itemize}
    \item Deep Reinforcement Opponent Network (DRON)
  \end{itemize}

\end{frame}


\begin{frame}[plain]


\only<4->{\vspace{-.5cm}}

  \begin{columns}[T]
    \column{.3\linewidth}

    \only<1->{ \includegraphics[width=2\linewidth]{qb/feature_ex_l_1} \\ }
    \vspace{.5cm}
    \only<4->{ \includegraphics[width=2\linewidth]{qb/feature_ex_l_2}  \\ }
    \vspace{.5cm}
    \only<7->{ \includegraphics[width=2\linewidth]{qb/feature_ex_l_3}  \\ }


    \column{.68\linewidth}
    \vspace{-.5cm}
    \only<2->{ \includegraphics[width=.85\linewidth]{qb/feature_ex_r_1} \\ }
    \only<3->{ \vspace{-.5cm} \hspace{.5cm} \includegraphics[width=.1\linewidth]{qb/feature_ex_wait}  \\ }
    \only<5->{ \includegraphics[width=\linewidth]{qb/feature_ex_r_2} \\ }
    \only<6->{ \vspace{-.5cm} \hspace{.5cm}\includegraphics[width=.1\linewidth]{qb/feature_ex_wait}  \\ }
    \only<8->{ \includegraphics[width=\linewidth]{qb/feature_ex_r_3} \\ }
    \only<9->{ \vspace{-.5cm} \hspace{.5cm} \includegraphics[width=.1\linewidth]{qb/feature_ex_buzz}  \\ }
    \only<9->{Answer: {\bf Julius Caesar}}
  \end{columns}

\end{frame}


\begin{frame}{Add more features: DRON-concat}

  \gfxq{dron-concat}{.8}

\end{frame}


\begin{frame}{Error Analysis}

  \only<1>{\gfxq{error1}{.8}}
  \only<2>{\gfxq{error2}{.8}}
  \only<3>{\gfxq{error3}{.8}}
  \only<4>{\gfxq{error4}{.8}}
  \only<5>{\gfxq{error5}{.8}}
  \only<6>{\gfxq{error6}{.8}}
  \only<7>{\gfxq{error7}{.8}}

\end{frame}



\begin{frame}{How the shared task works}

\begin{columns}
  \column{.3\linewidth}
  \gfxq{bamber}{.5}

  \column{.65\linewidth}
  \begin{itemize}
    \item<3-> Hi! Available questions are \texttt{[1,2,3,4]}
    \item<5-> It's \texttt{Extremism}
    \item<7-> It's \texttt{in}
    \item<9-> It's \texttt{the}
    \item<11-> Got it!  You've answered Question 1 at Position
      3 with \texttt{Barry\_Goldwater}
  \end{itemize}

\end{columns}


\begin{columns}

  \column{.65\linewidth}
  \begin{itemize}
    \item<2-> I'm User~1.  I’d like to play!
    \item<4-> I’d like to hear Word~1 of Question 1
    \item<6-> I’d like to hear Word~2 of Question 1
    \item<8-> I’d like to hear Word~3 of Question 1
    \item<10-> I’d like to answer Question 1 with
      \texttt{Barry\_Goldwater}
    \end{itemize}
  \column{.3\linewidth}
  \only<2->{\gfxq{buzzer}{.5}}
\end{columns}

\end{frame}


\begin{frame}{Are there enough opportunities?}


\begin{center}
\begin{tabular}{lcccc}
\toprule
& \alert<2>{verb} & \alert<3>{voice} & \alert<4>{noun} & \alert<5>{clauses} \\
\midrule
Applicable \% & \only<2->{39.9} & \only<3->{50.0} & \only<4->{26.4} & \only<5->{4.8} \\
Accepted \% & \only<2->{22.5} & \only<3->{24.0} & \only<4->{51.2} & \only<5->{38.4} \\
\bottomrule
\end{tabular}
\end{center}

\only<2>{
\begin{itemize*}
\item[O:] {\bf They announced} that the president will restructure the division.
\item[R:] The president will restructure the division, {\bf they announced}.
\end{itemize*}
}


\only<3>{
  \begin{columns}
    \column{.4\linewidth}

    \gfxs{rewrite_input}{.9}
    \column{.55\linewidth}
    \vspace{-.5cm}
    \gfxs{rewrite_transform}{.9}
   \end{columns}
}

\only<4>{
\begin{itemize*}
  \item[O:] the e-mail server of Clinton
  \item[R:] Clinton's e-mail server
\end{itemize*}
}

\only<5>{
\begin{itemize*}
\item[O:] \pos{S}$_1$ \pos{conj} \pos{S}$_2$: We should march {\bf because} winter is coming.
\item[O:] \pos{conj} \pos{S}$_2$, \pos{S}$_1$: {\bf Because} winter is
  coming, we should march.
\vspace{1cm}
\item[R:] \pos{S}$_2$, \pos{conj'} \pos{S}$_1$: Winter is coming, {\bf because of this}, we should march.
\end{itemize*}
}


\end{frame}

\begin{frame}{}

\gfxs{rewrite_eval}{.8}
\pause
We rewrite 32.2\% of
sentences, reducing the delay from 9.9 words/seg to 6.3 words/seg per
segment for rewritten sentences and from 7.8 words/seg to 6.7 words/seg overall.

\end{frame}

\begin{frame}{How good are the translations?}

  \gfxs{tradeoff-rw-bleu}{.7}

\begin{center}
Aggressiveness based on different right probability thresholds
\cite{fujita-13}
\end{center}

\end{frame}

\begin{frame}{Why does the quality improve?}

\begin{center}
\begin{tabular}{ccccc}
\toprule
& \multicolumn{3}{c}{Translation} & \\
\cmidrule{2-4}
& \abr{gd} & \abr{rw} & \abr{rw+gd} & Gold ref \\
\midrule
\# of verbs & 1971 & 2050 & {\bf 2224} & 2731 \\
\bottomrule
\end{tabular}
\end{center}

\end{frame}

\begin{frame}{Future Steps}

  \begin{itemize}
    \item Verb prediction through argument structure \only<2->{(entity
        / coref)}
    \item Richer translation model~\cite{oda-15,cho-16}
    \item Better reward (e.g., MEANT)  \only<3->{(richer answer space)}
    \item Paraphrase database
    \item Learning from imperfect feedback  \only<4->{(player clues)}
    \item Centaur translations  \only<5->{(centaur QB)}
  \end{itemize}

\end{frame}

\begin{frame}{Where we have problems}

\only<1-2>{
\begin{block}{Out of Date}
Although he won the California primary in 2000, he distanced himself
from fellow reform presidential candidate Pat Buchanan by comparing
him to Attila the Hun. After being called a jackass, he prompted
Lindsey Graham to destroy his phone by giving out his number during a
speech. The slogan (*) Make America Great Again has been used by this
politician, who claimed he didn't like people who were captured as a
slight to John McCain and kicked off his 2016 presidential bid with
some inflammatory remarks about Mexicans. For 10 points, name this
Republican candidate and real estate mogul.
\end{block} }
\only<2>{{\bf Chris Christie?}}

\only<3-4>{
\begin{block}{Out of Touch}
  This singer recently cancelled the Great Escape Tour, and, in one
  song, she claims that she will be ``Eating crumpets with the sailors
  / On acres without the neighbors.'' She collaborated with Jennifer
  (*) Hudson on the song ``Trouble,'' which was issued in her album
  update Reclassified. This artist of ``Change Your Life'' was
  inspired by scenes from the movie Clueless to make the music video
  for a song in which she collaborated with Charli XCX. For 10 points,
  name this Australian rapper whose album The New Classic contained
  ``Fancy.''
\end{block} }
\only<4>{{\bf Bruce Springsteen?}}

\end{frame}



\end{document}

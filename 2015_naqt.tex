 
\documentclass[compress]{beamer}

%\usepackage{beamerthemesplit}
\usepackage{xmpmulti}

\usepackage{graphicx,float,wrapfig, bbm}
\usepackage{amsfonts, bbold, comment}
\usepackage{mdwlist}
\usepackage{subfigure}
\usepackage{colortbl}

\usepackage{multirow}

\pgfdeclareimage[width=\paperwidth]{mybackground}{colorado/boulder.pdf}

\newcommand{\e}[2]{\mathbb{E}_{#1}\left[ #2 \right] }
\newcommand{\ind}[1]{\mathbb{I}\left[ #1 \right] }
\newcommand{\ex}[1]{\mbox{exp}\left\{ #1\right\} }
\newcommand{\g}{\, | \,}
\newcommand{\citename}[1]{#1 }

\newcommand{\gfxs}[2]{
\begin{center}
	\includegraphics[width=#2\linewidth]{simtrans/#1}
\end{center}
}

\newcommand{\gfxq}[2]{
\begin{center}
	\includegraphics[width=#2\linewidth]{qb/#1}
\end{center}
}


\usetheme[bullet=circle,                     % Use circles instead of squares for bullets.
          titleline=true,                    % Show a line below the frame title.
          showdate=true,                     % show the date on the title page
          alternativetitlepage=true,         % Use the fancy title page.
          titlepagelogo=general_figures/culogo,              % Logo for the first page.
          % Logo for the header on first page.
          headerlogo=general_figures/boulder_cs,
          ]{UCBoulder}

\usecolortheme{ucdblack}
\title[Thinking on Your Feet]{Thinking on your Feet: \\ Seattle Exhibition Match}
\author{ Jordan Boyd-Graber}
\date{October 2, 2015}

\institute[Boulder] % (optional, but mostly needed)
{University of Colorado Boulder}

\AtBeginSection[] % "Beamer, do the following at the start of every section"
{ \begin{frame} \frametitle{Outline} % make a frame titled "Outline"
\tableofcontents[currentsection] % show TOC and highlight current section
\end{frame} }

\begin{document}

\frame{
\titlepage
\tiny
}

\section{Introduction}

\begin{frame}{The Competition}

\begin{itemize}
	\item A computer that plays quiz bowl: \textsc{qanta}
	\item Facing off against Ken Jennings
	\pause
	\item But first
	\begin{itemize}
		\item Who we are
		\item How \textsc{qanta} works
                \item Connection to hard real-world problems
		\item Introducing our opponent
	\end{itemize}
\end{itemize}

\end{frame}

\section{QANTA}

\begin{frame}
	\frametitle{How is this different from Watson?}

	\begin{columns}
		\column{.5\linewidth}

		\includegraphics[width=1.0\linewidth]{qb/jeopardy}


		\column{.5\linewidth}
		\begin{itemize}
			\item This is {\bf not} Jeopardy \cite{ferruci-10}
			\item There are buzzers, but players only buzz
                          at the question end
			\item Doesn't discriminate knowledge
			\item Quiz bowl is pyramidal
                        \item Watson must decide to answer {\bf once}, after
                          complete question
                        \item Quiz Bowl: decide after each word
                        \pause
                        \item We're not \textsc{ibm}
		\end{itemize}

	\end{columns}

\end{frame}


\begin{frame}[t]
	\frametitle{Sample Question 1}

With Leo Szilard, he invented a doubly-eponymous \only<2->{refrigerator with no moving parts. He did not take interaction with neighbors into account when formulating his theory of} \only<3->{heat capacity, so} \only<4->{Debye adjusted the theory for low temperatures. His} \only<4->{summation convention automatically sums repeated indices in tensor products. His name is attached to the A and B coefficients} \only<5->{for spontaneous and stimulated emission, the subject of one of his multiple groundbreaking 1905 papers. He further developed the model of statistics sent to him by} \only<6->{Bose to describe particles with integer spin. For 10 points, who is this German physicist best known for formulating the} \only<7->{special and general theories of relativity?} \\
\vspace{1cm}
\only<8->{ {\bf Albert \underline{Einstein}}}

\end{frame}


\begin{frame}{Who we are}

	\begin{columns}
		\column{.5\linewidth}
			\begin{block}{Jordan Boyd-Graber}
			
			\begin{itemize}
				\item Professor at Colorado
				\item Former quiz bowl player at Caltech (4$^{th}$, 2004 UG ICT) and Princeton (4$^{th}$, ACF Nats 2005)
			\end{itemize}
			
			\end{block}
			\gfxq{jordan_qb}{.5}
		
		
		\column{.5\linewidth}
			\begin{block}{Mohit Iyyer}
			
			\begin{itemize}
				\item $n$-year PhD student, University of Maryland
				\item National Champion, 2008 HSNCT
			\end{itemize}
			
			\end{block}
			\gfxq{mohit_qb}{.5}	
	\end{columns}
        Other folks: Hal Daum\'e III, Anupam Guha, He He, Brianna
        Satinoff, Manjhunath Ravi, Danny Bouman
\end{frame}

\section{How QANTA works}

\begin{frame}{Two Steps to Answering Questions}
	Given a question:
	\begin{enumerate}
		\item Generate a set of {\bf guesses} (deep learning)
		\item Select the best guess and {\bf buzz} if confident enough (classifier)
			% \begin{itemize}
			% 	\item Most creative
			% 	\item Can be improved (more later)
			% \end{itemize}
	\end{enumerate}
\end{frame}

\begin{frame}{}

  \begin{columns}
    \column{.4\linewidth}
        \includegraphics[width=0.9\linewidth]{general_figures/mohit}
    \column{.6\linewidth}
        \begin{block}{ {\bf \href{http://cs.colorado.edu/~jbg//docs/2014_emnlp_qb_rnn.pdf}{A Neural Network for Factoid Question Answering over Paragraphs}}}
\underline{\href{http://cs.umd.edu/~miyyer/}{Mohit Iyyer}}, {\bf Jordan Boyd-Graber}, Leonardo Claudino, Richard Socher, and Hal {Daum\'{e} III}.  \emph{Empirical Methods in Natural Language Processing}, 2014
        \end{block}
        
	\begin{block}{ {\bf \href{http://cs.colorado.edu/~jbg//docs/2015_acl_dan.pdf}{Deep Unordered Composition Rivals Syntactic Methods for Text Classification}} }
	\underline{\href{http://cs.umd.edu/~miyyer/}{Mohit Iyyer}}, {\bf Jordan Boyd-Graber}, and Hal {Daum\'{e} III}.  \emph{Association for Computational Linguistics}, 2015
	
	\end{block}        
        
  \end{columns}
\end{frame}


\begin{frame}{Guesses: Vector Space Model}


  \only<1>{\gfxq{unigram_models_2}{.9}}
  \only<2>{\gfxq{unigram_models_3}{.9}}
  \only<3>{\gfxq{unigram_models_4}{.9}}
  \only<4>{\gfxq{unigram_models_5}{.9}}
  \only<5>{\gfxq{unigram_models_6}{.9}}
  \only<6>{\gfxq{unigram_models_7}{.9}}
  \only<7>{\gfxq{unigram_models_8}{.9}}


\end{frame}

\begin{frame}{Guesses: Vector Space Model}

  \gfxq{embedding}{1.0}

\end{frame}


\begin{frame}{Guesses Example}

	\begin{block}{Question}
	The family sees Stone Mountain and has barbequed sandwiches at The Tower, run by Red Sammy, then heads for Florida over the grandmother's objections.
	\end{block}
	
	\begin{columns}
		\column{.5\linewidth}
		\only<2->{
		\begin{itemize}
			\item Cormac McCarthy
			\item Love in the Time of Cholera
			\item Miguel Angel Asturias
			\item Chronicle of a Death Foretold
			\item \alert<3>{A Good Man Is Hard to Find (short story)}
			\item<2-> {\bf and 195 others!}
		\end{itemize}
		}
		
		\column{.5\linewidth}
		\only<3>{
		The right answer is in this set 80\% of the time.  How do we know which one is right?  And whether we should buzz?}
	\end{columns}

\end{frame}

\begin{frame}{Is a guess right?}

	\begin{columns}
	\column{.75\linewidth}
	\begin{block}{Text}
	The family sees Stone Mountain and has barbequed sandwiches at The Tower, run by Red Sammy, then heads for Florida over the grandmother's objections.
	\end{block}
	\column{.25\linewidth}
	\begin{block}{Guess}
	A Good Man is Hard to Find (short story)
	\end{block}
	\end{columns}
	
	\begin{itemize}
		\item ``Red Sammy'' appears in three questions with this answer and in no other questions
		\item ``grandmother's objections'' appears in the answer's Wikipedia page
		\item The guess does not appear in clues
		\item The question looks like it's about a work
	\end{itemize}
	
	\pause
	
	We use a classifier (Vowpal Wabbit) to decide when to {\bf buzz} or {\bf wait}

\end{frame}

\begin{frame}[t]{Features (by example)}

\only<4->{\vspace{-1.5cm}}

  \begin{columns}[T]
    \column{.3\linewidth}

    \only<1->{ \includegraphics[width=2\linewidth]{qb/feature_ex_l_1} \\ }
    \vspace{.5cm}
    \only<4->{ \includegraphics[width=2\linewidth]{qb/feature_ex_l_2}  \\ }
    \vspace{.5cm}
    \only<7->{ \includegraphics[width=2\linewidth]{qb/feature_ex_l_3}  \\ }


    \column{.68\linewidth}
    \vspace{-.5cm}
    \only<2->{ \includegraphics[width=.85\linewidth]{qb/feature_ex_r_1} \\ }
    \only<3->{ \vspace{-.5cm} \hspace{.5cm} \includegraphics[width=.1\linewidth]{qb/feature_ex_wait}  \\ }
    \only<5->{ \includegraphics[width=\linewidth]{qb/feature_ex_r_2} \\ }
    \only<6->{ \vspace{-.5cm} \hspace{.5cm}\includegraphics[width=.1\linewidth]{qb/feature_ex_wait}  \\ }
    \only<8->{ \includegraphics[width=\linewidth]{qb/feature_ex_r_3} \\ }
    \only<9->{ \vspace{-.5cm} \hspace{.5cm} \includegraphics[width=.1\linewidth]{qb/feature_ex_buzz}  \\ }
    \only<9->{Answer: {\bf Julius Caesar}}
  \end{columns}

\end{frame}



\begin{frame}{It's not all fun and games \dots}

  \begin{columns}
    \column{.4\linewidth}
    \begin{center}
        \includegraphics[width=0.7\linewidth]{general_figures/hehe} \\
        \includegraphics[width=0.7\linewidth]{general_figures/alvin}
      \end{center}
    \column{.6\linewidth}
        \begin{block}{ {\bf \href{http://cs.colorado.edu/~jbg/docs/2015_emnlp_rewrite.pdf}{Syntax-based Rewriting for Simultaneous Machine Translation}}}
He He, Alvin Grissom II, {\bf Jordan Boyd-Graber}, and Hal {Daum\'{e} III}.  \emph{Empirical Methods in Natural Language Processing}, 2015
        \end{block}
        
\begin{block}{ {\bf \href{http://cs.colorado.edu/~jbg/docs/2014_emnlp_simtrans.pdf}{Don't Until the Final Verb Wait: Reinforcement Learning for Simultaneous Machine Transl\
ation}}}
\underline{\href{http://www.umiacs.umd.edu/~alvin/}{Alvin Grissom II}}, {\bf Jordan Boyd-Graber}, He He, John Morgan, and Hal {Daum\'{e} III}.  \emph{Empirical Methods in Natural L\
anguage Processing}, 2014
        \end{block}        
  \end{columns}
\end{frame}

\begin{frame}{Why simultaneous translation hard is}

  \begin{columns}
    \column{.5\linewidth}
       \gfxs{nuremberg_translators}{.9}
    \column{.5\linewidth}
       \begin{itemize}
         \item Languages like German and Japanese are {\bf verb final}
         \item Simultaneous translation requires you to think on your feet
         \item Predict when you know the verb so you can translate to English
       \end{itemize}
  \end{columns}

\end{frame}

\section{Competition}

\begin{frame}{Inconclusive First Steps}

		\begin{columns}
			\column{.25\linewidth}
				\gfxq{colby_jeo}{1.0} 
                                Colby Burnett:
                                \$375,000
			\column{.25\linewidth}
				\gfxq{ben_jeo}{1.0}
                                Ben Ingram:
                                \$427,534
			\column{.25\linewidth}
				\gfxq{alex_jeo}{1.0}
                                Alex Jacobs: \$151,802
			\column{.25\linewidth}
				\gfxq{kristin_jeo}{1.0}
                                Kristin Sausville: \$95,201
		\end{columns}

                \pause

                \begin{center}
                End result: 200-200 tie!
                \end{center}

\end{frame}


\begin{frame}{What's next}

	\begin{itemize}
		\item 40 questions written by Kurtis Droge
                  \item Read by Christopher Grubb
		\item \textsc{qanta} sees a word when I push a button: decides when to answer
		\pause

                \item Our opponent
                  \gfxq{jennings}{.5}
	\end{itemize}

\end{frame}




\begin{frame}{Questions and Discussion}

	\begin{center}
	\begin{Huge}
	?
	\end{Huge}
	\end{center}

\end{frame}


\begin{frame}{Postmortem}

\begin{itemize}
	\item Only most common 5,000 answers
	\item Trash and current events: too much churn
	\item Creative questions
	\item Common link questions
	\item Before and after
	\item Wordplay
	\item Computation
\end{itemize}


\end{frame}

\begin{frame}{We want (and need) your help!}

	\begin{itemize}
		\item Our system isn't perfect
		\item We need more data
		\item We need more features
		\item We need excellent coders	
	\end{itemize}
	
	\pause

	\begin{block}{Find out more \dots}
		\begin{itemize}
			\item Code: \url{http://github.com/miyyer/qb}
			\item Twitter: @boydgraber
			\item Announcements on HSQB
		\end{itemize}
	\end{block}

\end{frame}

\begin{frame}{Come to Boulder}

\begin{columns}
	\column{.5\linewidth}
        \only<1>{
        	\begin{center}
		\includegraphics[width=.9\linewidth]{colorado/boulder} \\
		\includegraphics[width=.9\linewidth]{colorado/cs_dept}
	\end{center}
              }
	\column{.5\linewidth}	
		\begin{itemize}
                \item Looking for undergrads/grads/interns
                \item A great place for natural language
                  processing (but no UW)
		\end{itemize}
\end{columns}

\end{frame}


\frame{

	\frametitle{Thanks}

        \begin{block}{Collaborators}
          \textsc{naqt}, Hal Daum\'e III (UMD), Anupam Guha (Maryland), Manjhunath Ravi (Colorado), Danny Bouman (UMD UG),
          Stephanie Hwa (UMD UG)
        \end{block}

	\begin{columns}
	
	\column{.3\linewidth}
        \begin{block}{Funders}
        \begin{center}
          \includegraphics[width=0.4\linewidth]{general_figures/nsf}
       \end{center}
        \end{block}
        
	\column{.3\linewidth}        
        \begin{block}{Supporters}
        	\gfxq{naqt}{.4}
        \end{block}
        \column{.3\linewidth}
        \begin{block}{Host}
        	\includegraphics[width=.8\linewidth]{general_figures/uw}
        \end{block}        

        \end{columns}
}



\begin{frame}{References}
\bibliographystyle{style/acl}
\tiny
\bibliography{bib/journal-full,bib/jbg}
\end{frame}




	%

\begin{frame}
	\frametitle{Learning which Features are Useful}

	\begin{itemize}
		\item Use how humans these data as a prior for supervised maxent model~\cite{daume-04}
		\item Prior for label $a$ and feature $f$ is a function of the number of buzzes $b$ and tf-idf~\cite{salton-68}
\begin{equation}
  \left[ \vphantom{\frac{a}{b}}\alpha \alert<4>{\ind{ b(a,f) > 0}} + \beta \alert<3>{ b(a,f)} + \gamma
  \right] \alert<2>{\mbox{tf-idf}(a,f)} .
\label{eq:meanweight}
\end{equation}
		\begin{itemize}
			\item $\alpha$, $\beta$, and $\gamma = 0$: na\"ive zero prior
			\item $\alpha$ and $\beta = 0$: linear transformation of the mean
			\item $\alpha$ and $\gamma = 0$: number of buzzes times tf-idf value of the features
		\end{itemize}

	\end{itemize}

\end{frame}

\begin{frame}
	\frametitle{Using buzzes as a prior}

\begin{equation*}
  \left[ \vphantom{\frac{a}{b}}\alpha \ind{ b(a,f) > 0} + \beta b(a,f) + \gamma
  \right] \mbox{tf-idf}(a,f) .
\end{equation*}

\begin{center}
\begin{tabular}{cccccc}
Answers & Weighting & $\alpha$ & $\beta$ & $\gamma$ & Error\footnote{Buzz and tf-idf computed on training data; grid search on dev data; error on test data} \\
\hline
\multirow{5}{*}{100} & zero & - & - & - & 0.22 \\
& tf-idf & - & - & 8.3 & 0.08 \\
&  buzz-binary & 10.7 & - & - & {\bf 0.06} \\
&  buzz-linear & - &  1.1 & - & 0.10 \\
& buzz-tier & - & 1.6 & 0.5 & 0.07 \\
\hline
\end{tabular}
\end{center}
\end{frame}





\begin{frame}
	\begin{center}

\vspace{-.6cm}
\begin{figure}[tb]
\centering

\subfigure[Buzzes over all Questions]{
\includegraphics[width=0.6\linewidth]{qb/buzz_cloud}
\label{fig:buzz_cloud}
}

\subfigure[Wuthering Heights Question Text]{
\includegraphics[width=0.45\linewidth]{qb/wuthering_heights_question}
\label{fig:wh_question}
}
\subfigure[Buzzes on Wuthering Heights]{
\includegraphics[width=0.45\linewidth]{qb/wuthering_heights_buzz}
\label{fig:wh_buzz}
}
\end{figure}


	\end{center}

\end{frame}



\begin{frame}
	\frametitle{Accuracy vs. Speed}

	\begin{center}
	  \includegraphics[width=0.8\linewidth]{qb/accuracy_vs_speed}
	  \end{center}

\end{frame}




\end{document}
